\section{Характеристика района похода}\label{sec:characteristics_of_area}
    Поход проходил в районе Центрального Кавказа (Безенги, Адырсу). Безенги (в переводе с Балкарского~"---
    <<место, откуда сошёл ледник>>)~"--- очень красивый, достаточно высокий по меркам Кавказа район. Включает
    Безенгийскую стену главного Кавказского хребта и прилегающие с севера боковые хребты с ледником Безенги
    (бассейн реки Черек Безенгийский). Главная достопримечательность~"--- Безенгийская стена, грандиозный хребет
    разделяющий Россию и Грузию, ее протяженность 12\,км и она включает в себя следующие вершины: Шхара Главная,
    5068\,м~"--- Западная Шхара, 5057\,м~"--- пик Шота Руставели, 4960\,м~"--- Джанги-тау Главная, 5085\,м~"---
    Катын, 4974\,м~"--- Гестола, 4860\,м~"--- Ляльвер, 4350\,м~"--- пик\\ <<4310>>. При планировании маршрута мы
    также выяснили, что специалистами ФГБУ <<Высокогорного Геофизического института>> в результате анализа
    разновременных космоснимков было выявлено, что в начале мая 2023 года на участке Безенгийской стены под горой
    Джангитау Западная (5059\,м) образовалась трещина, которая непрерывно увеличивается. Специалисты опасаются, что
    в скором времени произойдет обрушение блока льда с захватом подстилающих пород и не рекомендуют останавливаться
    на Судейских и Угловых ночевках.

    Долины рек здесь часто узкие, труднопроходимые. Один из наиболее протяженных и живописных каньонов~"--- каньон реки
    Дыхсу, который мы обходили через перевалы туристов Грузии и Ашинова. На высотах до 2500\,м часто можно встретить леса,
    и чем выше забираешься, тем лучше видно как хвойные породы деревьев постепенно сменяются лиственными и ближе к границе
    леса встречаются в основном низкорослые березы. Вдоль русла рек часто можно встретить труднопроходимые заросли рододендрона
    и ивы. В лесах водятся медведи, волки, зайцы, но мы никого из животных не встретили. Следует отметить, что в связи с
    глобальным потеплением ледники тают, становятся более разорванными, однако в июне еще остаются надежные снежные мосты.
    Скалы в районе преимущественно разрушенные, многие перевалы и вершины к августу становятся камнеопасными. 

    Если говорить о климате этого района, довольно часто в первой половине дня погода хорошая, потом поднимаются облака,
    приходит туман и дождь или снег. Однако встречаются как и совсем дождливые дни с плохой видимостью с самого утра, так
    и ясные жаркие дни (но их значительно меньше). Во время похода полностью ясных дней было 2 из 19, около 4 дней с плохой
    погодой с самого утра.

    Безенгийское ущелье находится в пограничной зоне и на его территории действует пропускной режим. Удобно заезжать в район по долине реки Черек Балкарский (здесь находится т/б Уштулу) или по долине реки
    Черек Безенгийский (здесь находится альплагерь Безенги). Тропы в районе проходят там, где есть альпинистские или простые
    треккинговые маршруты.

    Район Адырсу больше похож на типичный Центральный Кавказ, долины рек здесь широкие, часто можно встретить крупный рогатый скот. Заезд в
    район легко осуществить через турбазу Башиль или альплагерь Уллутау. Часто можно встретить туристов и альпинистов,
    достаточно много троп. С перевалов можно увидеть Безенгийский массив, Эльбрус, массив Уллутау.
