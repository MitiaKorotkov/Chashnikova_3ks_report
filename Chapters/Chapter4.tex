\section{Подробная информация о маршруте}\label{sec:route}
	\subsection{Изменения на маршруте и их причины}\label{subsec:changes_of_way}
1) отказ от радиального восхождения на пер. Песчаный с лед. Крумкол, из-за отсутствия мотивации. Эстетические потребности группа закрыла на пер. Ашинова и больше нужды в них не было. Вместо радиалки на пер. Песчаный осуществили разведку ледопада на пути к пер. Спартак
2) отказ от ледовой тренировки, из-за отсутствия запаса по времени и тенденции к ухудшению погоды во второй половине дня 
3) внеплановая дневка в а/л Безенги из-за накопленной усталости на десятый день похода
4) отказ от радиального восхождения на п. МВТУ из-за накопленной усталости, отсутствия запаса по времени и плохого прогноза погоды (дождь)
5) отказ от прохождения траверса пер. Кенчат+Кенчатбаши+Килар (2А) и прохождение пер. Килар насквозь (1Б) + радиальный выход в миникольцо: пер. Надежда (1Б, п/п) + пер. Кенчат + вер. Кенчатбаши + траверс гребня почти до пер. Килар (2А) (не дошли 100 м по гребню, сбросили раньше). В связи с накопленным отставанием на 1 день по план-графику было решено сократить кольцо, заменив траверс малоизвестных препятствий на запасной вариант. Это позволило сэкономить 1 день, а также мы могли бы отказаться от радиалки при плохой погоде. Однако погода на 3 кольце установилась хорошая и мы успели сходить на запланированные препятствия.

	
	\subsection{График движения на маршруте}
		\newcolumntype{A}{ >{\centering\arraybackslash} m{0.4cm} }
		\newcolumntype{B}{ >{\centering\arraybackslash} m{1.4cm} }
		\newcolumntype{C}{ >{\centering\arraybackslash} m{5.0cm} }
		\newcolumntype{D}{ >{\centering\arraybackslash} m{1.3cm} }
		\newcolumntype{E}{ >{\centering\arraybackslash} m{4.5cm} }
		\setTextVar{\textOne}{При радиальных выходах километраж учитывается при движении лишь в одну сторону. В таблице в скобках указан километраж, без учёта повторно пройденного пути. При этом перепад высот, преодолённый на радиальных выходах, учитывается полностью.}
		\setTextVar{\textTwo}{Это километраж радиального выхода, совершённого частью группы}
		{\footnotesize
		\begin{longtable}{|A|B|C|B|B|D|E|} \hline
			№							&	Дата 		&	Участок пути																																																&	Путь, км\footnote{\textOne}	&	Перепад высот, м		&	ЧХВ		&	Характер пути																\\ \hline
			\hyperref[subsec:Day1]{1}	&	30.06.24	&	Т/б~Штуллу~--- лед.~Штуллу																																													&	12.0						&	$+850$ $$			&			&	Дорога, тропа																\\ \hline
			\hyperref[subsec:Day2]{2}	&	01.07.24	&	Лед.~Штуллу~---\hyperref[subsec:main_obstacles]{пер.~Штулу + п.~Штулу~(1А, рад.)}~--- т/б~Штуллу																											&	20.1 ( в зачет 3,9)						&	$+1000$ $-1950$			&			&	Тропа, дорога																\\ \hline
			\hyperref[subsec:Day3]{3}	&	02.07.24	&	Т/б~Штуллу~--- д.р.~Черек Балкарский~--- д.р.~Тютюн	су																																						&	11.5						&	$+950$ $-270$			&			&	Дорога, тропа, лес, заросли кустарника, курумник							\\ \hline
			\hyperref[subsec:Day4]{4}	&	03.07.24	&	Д.р.~Тютюнсу~--- \hyperref[subsec:main_obstacles]{пер.~Туристов Грузии~(1Б)}~--- ночёвки в цирке пер.~Туристов Грузии																						&	4.4							&	$+1140$ $-230$ 			&			&	Травянистые склоны, морены, осыпь, открытый ледник, снег					\\ \hline
			\hyperref[subsec:Day5]{5}	&	04.07.24	&	Ночёвки в цирке пер.~Туристов Грузии~--- \hyperref[subsec:main_obstacles]{пер.~Ашинова~(1Б)}~--- лед.~Крумкол																								&	6.0							&	$+550$ $-500$ 			&			&	Осыпь, открытый ледник, снег												\\ \hline
			\hyperref[subsec:Day6]{6}	&	05.07.24	&	Лед.~Крумкол~--- \hyperref[subsec:main_obstacles]{пер.~Спартак~(2А)}																																		&	4.0							&	$+650$ 			 		&			&	Ледопад, закрытый ледник													\\ \hline
			\hyperref[subsec:Day7]{7}	&	06.07.24	&	\hyperref[subsec:main_obstacles]{Пер.~Спартак + п.~Башхауз + пер.~МВТУ~(2А)}																																&	1.5							&	$+350$ $-270$	 		&			&	Разрушенные скалы, осыпь, закрытый ледник									\\ \hline
			\hyperref[subsec:Day8]{8}	&	07.07.24	&	Лед.~Безенги~--- ночёвки Баран-кош																																											&	11.6						&	 $-1500$	 		&			&	Закрытый разорванный ледник, открытый ледник, тропа							\\ \hline
			\hyperref[subsec:Day9]{9}	&	08.07.24	&	Ночёвки Баран-кош~--- а/л~Безенги																																											&	7.8							&	 $-650$	 		&			&	Открытый ледник, тропа														\\ \hline
			\hyperref[subsec:Day10]{10}	&	09.07.24	&	А/л~Безенги~--- ночёвки под пер.~Столбовой 																																									&	3.5							&	$+1200$ $-100$	 		&			&	Травянистые склоны															\\ \hline
			\hyperref[subsec:Day11]{11}	&	10.07.24	&	Ночёвки под пер.~Столбовой~--- \hyperref[subsec:main_obstacles]{пер.~Столбовой~(2А)}~--- лед.~Кору 																											&	6.7							&	$+770$ $-710$	 		&			&	Курумник, морены, открытый ледник											\\ \hline
			\hyperref[subsec:Day12]{12}	&	11.07.24	&	Лед.~Кору~--- \hyperref[subsec:main_obstacles]{пер.~Тютюргу~(1Б)}~--- лед.~Тютюргу~--- \hyperref[subsec:main_obstacles]{пер.~Шаурту~(2А)}																	&	5.4							&	$+1060$ $-340$	 		&			&	Открытый ледник, закрытый ледник, лифтовая осыпь, скалы						\\ \hline
			\hyperref[subsec:Day13]{13}	&	12.07.24	&	\hyperref[subsec:main_obstacles]{Пер.~Шаурту~(2А)}~--- лед.~Шаурту 																																			&	7.7							&	$+85$ $-1710$	 		&			&	Лифтовая осыпь, закрытый ледник, травянистые склоны, морены					\\ \hline
			\hyperref[subsec:Day14]{14}	&	13.07.24	&	Лед.~Шаурту~--- д.р.~Тютюргу~--- д.р.~Гара-Аузу-Су~--- д.р.~Башиль-Аузу-Су~--- т/б~Башиль																													&	17.6						&	$+250$ $-540$	 		&			&	Травянистые склоны, лес, дорога, тропа										\\ \hline
			\hyperref[subsec:Day15]{15}	&	14.07.24	&	Т/б~Башиль~--- ночёвки под пер.~Килар																																										&	8.4							&	$+1480$ 		 		&			&	Тропа, травянистые склоны, средняя осыпь, морены							\\ \hline
			\hyperref[subsec:Day16]{16}	&	15.07.24	&	Ночёвки под пер.~Килар~--- \hyperref[subsec:main_obstacles]{пер.~Кенчат + п.~Кенчатбаши~(2А, рад.)}~--- \hyperref[subsec:main_obstacles]{пер.~Килар~(1Б)}~--- лед.~Кенчат Западный~--- ночевки Тютю нижние	&	11,2 (в зачет 9,5, из низ 3,5 рад.) \footnote{\textTwo}	&	$950$ $-1500$	&			&	Фирн, снежные склоны, подвижная осыпь, закрытый ледник, морены				\\ \hline
			\hyperref[subsec:Day17]{17}	&	16.07.24	&	Ночевки Тютю нижние~--- \hyperref[subsec:main_obstacles]{пер.~Тютю Зап.~(2А)}																																&	3.5							&	$+1180$					&			&	Тропа, закрытый ледник, снежные и ледовые склоны							\\ \hline
			\hyperref[subsec:Day18]{18}	&	17.07.24	&	\hyperref[subsec:main_obstacles]{Пер.~Тютю Зап. + п.~Тютюбаши 2-ая Зап. + пер.~Куллумкол + пер.~Шогенцукова~(2А)}~--- д/р~Куллумкол-Су~--- а/л~Джайлык														&	10.3 (в зачет 10)						&	$+100$ $-2000$		 	&			&	Скалы, подвижная осыпь, открытый ледник, морены, травянистые склоны, тропа	\\ \hline
			\hyperref[subsec:Day19]{19}	&	18.07.24	&	А/л~Джайлык~--- Верхний Баксан																																												&	9.5							&	$-800$				 	&			&	Дорога																		\\ \hline
			\multicolumn{3}{|c|}{ИТОГО}																																																									&	163.1						&	$+13490$ $-13840$		&			&																				\\ \hline
		\end{longtable}
		}
	
	
	\subsection{Карта маршрута}
		\mbox{\includegraphics[scale=0.55, angle=90]{Pictures/Chapter4/Map.pdf}}

		Итого активными способами передвижения группой пройдено\footnote{Километраж, преодолённый на самолёте,
		микроавтобусе и т.д., в таблице не приводится. Зачётный километраж приведён с коэффициентом k = 1,1.}:
		\textbf{180 км} за \textbf{19 ходовых дней}, из них в зачёт: \textbf{159,4 км}.

		Трек похода доступен по \href{https://nakarte.me/#m=10/43.08945/43.14674&l=O&nktl=fI_Vhwot_mXwo3snYI90KA}{ссылке},
		а также в прикреплённом \href{run:./track.gpx}{файле gpx}.
	
	
	\subsection{Высотный профиль маршрута}
		\mbox{\includegraphics[width=\textwidth]{Pictures/Chapter4/profile.pdf}}
