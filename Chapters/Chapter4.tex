\section{Подробная информация о маршруте}\label{sec:route}
	\subsection{Изменения на маршруте и их причины}\label{subsec:changes_of_way}
		\begin{enumerate}
			\item Отказ от радиального восхождения на пер.~Песчаный с лед.~Крумкол, из-за отсутствия мотивации.
			Эстетические потребности группа закрыла на пер.~Ашинова. Вместо радиалки на пер.~Песчаный осуществили
			разведку потенциально непростого ледопада на пути к пер.~Спартак.
			\item Внеплановая полудневка в а/л~Безенги из-за накопленной усталости на девятый и десятый день похода.
			\item Отказ от радиального восхождения на п.~МВТУ из-за накопленной усталости, отсутствия запаса по
			времени и плохого прогноза погоды на вторую половину дня.
			\item Отказ от прохождения траверса пер.~Кенчат + Кенчатбаши + Килар (2А) и прохождение вместо этого
			пер.~Килар насквозь (1Б) + радиальный выход-миникольцо: пер.~Надежда (1Б, п/п) + пер.~Кенчат +
			вер.~Кенчатбаши + траверс гребня почти до пер.~Килар (2А) (не дошли 100\,м по гребню, спустились раньше).
			В связи с накопленным отставанием на 1 день по план-графику было решено сократить кольцо, заменив траверс
			малоизвестных потенциально камнеопасных препятствий на запасной и более надежный вариант. Это позволило
			сэкономить 1 день, а также мы могли бы отказаться от радиалки при плохой погоде. Однако погода на 3
			кольце установилась хорошая, сходов камней утром в середине июля не наблюдалась, и мы успели сходить на
			запланированные препятствия.
		\end{enumerate}
	
	
	\subsection{График движения на маршруте}
		\setlength{\tabcolsep}{2pt}
		
		\newcolumntype{A}{ >{\centering\arraybackslash} m{0.4cm} }
		\newcolumntype{B}{ >{\centering\arraybackslash} m{0.8cm} }
		\newcolumntype{C}{ >{\centering\arraybackslash} m{5.7cm} }
		\newcolumntype{D}{ >{\centering\arraybackslash} m{1.3cm} }
		\newcolumntype{E}{ >{\centering\arraybackslash} m{4.4cm} }
		\newcolumntype{F}{ >{\centering\arraybackslash} m{1.8cm} }
		\newcolumntype{K}{ >{\centering\arraybackslash} m{1.5cm} }
		\setTextVar{\textOne}{При радиальных выходах километраж учитывается при движении лишь в одну сторону. В таблице в
		скобках указан километраж, без учёта повторно пройденного пути. При этом перепад высот, преодолённый на радиальных
		выходах, учитывается полностью.}
		\setTextVar{\textTwo}{Это километраж радиального выхода, совершённого частью группы}
		{\footnotesize
		\begin{longtable}{|A|B|C|F|K|K|D|E|} \hline
			№	&	Дата						 	&	Участок пути																																																	&	Путь, км\footnote{\textOne}								&	$\Delta h$, м		&	Высота ноч., м		&	ЧХВ	за день	&	Характер пути																\\ \hline
			1 	&	\hyperref[subsec:Day1]{30.06}	&	Т/б~Штулу~"--- лед.~Штулу																																														&	12.0													&	$+850$ 				&	$2663$				&	4:42		&	Дорога, тропа																\\ \hline
			2 	&	\hyperref[subsec:Day2]{01.07}	&	Лед.~Штулу~"---\hyperref[subsec:main_obstacles]{пер.~Штулу + п.~Штулу~(1А, рад.)}~"--- т/б~Штулу																												&	20.1 (в зачет 3,9)										&	$+1000$ $-1950$		&	$1848$				&	8:26		&	Тропа, дорога																\\ \hline
			3 	&	\hyperref[subsec:Day3]{02.07}	&	Т/б~Штулу~"--- д.р.~Черек Балкарский~"--- д.р.~Тютюн	су																																						&	11.5													&	$+950$ $-270$		&	$2517$				&	8:10		&	Дорога, тропа, лес, заросли кустарника, курумник							\\ \hline
			4 	&	\hyperref[subsec:Day4]{03.07}	&	Д.р.~Тютюнсу~"--- \hyperref[subsec:main_obstacles]{пер.~Туристов Грузии~(1Б)}~"--- ночёвки в цирке пер.~Туристов Грузии																							&	4.4														&	$+1140$ $-230$ 		&	$3428$				&	6:24		&	Травянистые склоны, морены, осыпь, открытый ледник, снег					\\ \hline
			5 	&	\hyperref[subsec:Day5]{04.07}	&	Ночёвки в цирке пер.~Туристов Грузии~"--- \hyperref[subsec:main_obstacles]{пер.~Ашинова~(1Б)}~"--- лед.~Крумкол																									&	6.0														&	$+550$ $-500$ 		&	$3453$				&	6:10		&	Осыпь, открытый ледник, снег												\\ \hline
			6 	&	\hyperref[subsec:Day6]{05.07}	&	Лед.~Крумкол~"--- \hyperref[subsec:main_obstacles]{пер.~Спартак~(2А)}																																			&	4.0														&	$+650$ 			 	&	$4091$ 				&	4:32		&	Ледопад, закрытый ледник													\\ \hline
			7 	&	\hyperref[subsec:Day7]{06.07}	&	\hyperref[subsec:main_obstacles]{Пер.~Спартак + п.~Башхауз + пер.~МВТУ~(2А)}																																	&	1.5														&	$+350$ $-270$	 	&	$4168$ 				&	10:05		&	Разрушенные скалы, осыпь, закрытый ледник									\\ \hline
			8 	&	\hyperref[subsec:Day8]{07.07}	&	Лед.~Безенги~"--- ночёвки Баран-кош																																												&	11.6													&	$-1500$	 			&	$2755$				&	6:35		&	Закрытый разорванный ледник, открытый ледник, тропа							\\ \hline
			9 	&	\hyperref[subsec:Day9]{08.07}	&	Ночёвки Баран-кош~"--- а/л~Безенги																																												&	7.8														&	$-650$	 			&	$2146$				&	2:22		&	Открытый ледник, тропа														\\ \hline
			10	&	\hyperref[subsec:Day10]{09.07}	&	А/л~Безенги~"--- ночёвки под пер.~Столбовой 																																									&	3.5														&	$+1200$ $-100$	 	&	$3236$ 				&	5:39		&	Травянистые склоны															\\ \hline
			11	&	\hyperref[subsec:Day11]{10.07}	&	Ночёвки под пер.~Столбовой~"--- \hyperref[subsec:main_obstacles]{пер.~Столбовой~(2А)}~"--- лед.~Кору 																											&	6.7														&	$+770$ $-710$	 	&	$3296$ 				&	9:09		&	Курумник, морены, открытый ледник											\\ \hline
			12	&	\hyperref[subsec:Day12]{11.07}	&	Лед.~Кору~"--- \hyperref[subsec:main_obstacles]{пер.~Тютюргу~(1Б)}~"--- лед.~Тютюргу~"--- \hyperref[subsec:main_obstacles]{пер.~Шаурту~(2А)}																	&	5.4														&	$+1060$ $-340$	 	&	$4018$ 				&	7:26		&	Открытый ледник, закрытый ледник, лифтовая осыпь, скалы						\\ \hline
			13	&	\hyperref[subsec:Day13]{12.07}	&	\hyperref[subsec:main_obstacles]{Пер.~Шаурту~(2А)}~"--- лед.~Шаурту 																																			&	7.7														&	$+85$ $-1710$	 	&	$2390$ 				&	5:40		&	Лифтовая осыпь, закрытый ледник, травянистые склоны, морены					\\ \hline
			14	&	\hyperref[subsec:Day14]{13.07}	&	Лед.~Шаурту~"--- д.р.~Тютюргу~"--- д.р.~Гара-Аузу-Су~"--- д.р.~Башиль-Аузу-Су~"--- т/б~Башиль																													&	17.6													&	$+250$ $-540$	 	&	$2104$ 				&	4:41		&	Травянистые склоны, лес, дорога, тропа										\\ \hline
			15	&	\hyperref[subsec:Day15]{14.07}	&	Т/б~Башиль~"--- ночёвки под пер.~Килар																																											&	8.4														&	$+1480$ 		 	&	$3574$ 				&	5:51		&	Тропа, травянистые склоны, средняя осыпь, морены							\\ \hline
			16	&	\hyperref[subsec:Day16]{15.07}	&	Ночёвки под пер.~Килар~"--- \hyperref[subsec:main_obstacles]{пер.~Кенчат + п.~Кенчатбаши~(2А, рад.)}~"--- \hyperref[subsec:main_obstacles]{пер.~Килар~(1Б)}~"--- лед.~Кенчат Западный~"--- ночевки Тютю нижние	&	11,2 (в зачет 9,5, из них 3,5 рад.)\footnote{\textTwo}	&	$+950$ $-1500$		&	$3023$				&	10:17		&	Фирн, снежные склоны, подвижная осыпь, закрытый ледник, морены				\\ \hline
			17	&	\hyperref[subsec:Day17]{16.07}	&	Ночевки Тютю нижние~"--- \hyperref[subsec:main_obstacles]{пер.~Тютю Зап.~(2А)}																																	&	3.5														&	$+1180$				&	$4186$				&	5:35		&	Тропа, закрытый ледник, снежные и ледовые склоны							\\ \hline
			18	&	\hyperref[subsec:Day18]{17.07}	&	\hyperref[subsec:main_obstacles]{Пер.~Тютю Зап. + п.~Тютюбаши 2-ая Зап. + пер.~Куллумкол + пер.~Шогенцукова~(2А)}~"--- д/р~Куллумкол-Су~"--- а/л~Джайлык														&	10.3 (в зачет 10)										&	$+100$ $-2000$		&	$2309$				&	6:40		&	Скалы, подвижная осыпь, открытый ледник, морены, травянистые склоны, тропа	\\ \hline
			19	&	\hyperref[subsec:Day19]{18.07}	&	А/л~Джайлык~"--- Верхний Баксан																																													&	9.5														&	$-800$				&	$-$					&	2:06		&	Дорога																		\\ \hline
			\multicolumn{3}{|c|}{ИТОГО в зачёт}																																																							&	144.5 (159 с $k = 1,1$)									&	$+12550$ $-13070$	&						&				&																				\\ \hline
		\end{longtable}
		}
		\setlength{\arraycolsep}{5pt}
	
	\subsection{Карта маршрута}
		\hbox to \textwidth{%
		\hfil\includegraphics[height=0.9\textwidth, angle=90]{Pictures/Chapter4/Map.pdf}\hfil}

		Итого активными способами передвижения группой пройдено\footnote{Километраж, преодолённый на самолёте,
		микроавтобусе и т.д., в таблице не приводится. Зачётный километраж приведён с коэффициентом $k = 1,1$.}:
		\textbf{173{,}9 км} за \textbf{19 ходовых дней}, из них в зачёт: \textbf{158{,}4 км}.

		Трек похода доступен по \href{https://nakarte.me/#m=10/43.08945/43.14674&l=O&nktl=fI_Vhwot_mXwo3snYI90KA}{ссылке},
		а также в прикреплённом \textattachfile{./track.gpx}{файле gpx}.
	
	
	\subsection{Высотный профиль маршрута}
	% можно перфекционировать с размерами, я пока остановился на таком.
		\vspace{1cm}

		\hbox to \textwidth{%
		\hfil\includegraphics[width=1.04\textwidth]{Pictures/Chapter4/profile.png}\hfil}
