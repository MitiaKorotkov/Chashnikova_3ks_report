\graphicspath{{Pictures/Chapter5/Day18}}

\subsection{День 18 (17 июля)}\label{subsec:Day18}
    \vbox{%
        \hbox to \textwidth{\hfil%
        \includegraphics[width=0.8\textwidth]{day18.pdf}\label{fig:Day18_map}\hfil}
        \vspace{0.3cm}

        \hbox to \textwidth{\hfil%
        \begin{tabular}{|p{4.5cm}|>{\centering\arraybackslash}p{2.8cm}|}
            \hline
            Расстояние, км		&   $10,3$          \\
            Из них в зачёт, км  &   $10$            \\
            Набор высоты, м		&   $+100 / -2000$  \\
            Высота ночёвки, м	&   $2309$          \\
            \hline
        \end{tabular}\quad%
        \begin{tabular}{|p{7cm}|>{\centering\arraybackslash}p{1.5cm}|}
            \hline
            ЧХВ за день                                             &   6:40    \\
            ГХВ за день                                             &   8:05    \\
            \hline
            ЧХВ рад. выхода                                         &   00:35   \\
            ЧХВ от МН до пер.\,Куллумкол                            &   1:25	\\
            ЧХВ от пер.\,Куллумкол \newline до пер.\,Шогенцукова    &   1:05    \\
            ЧХВ от пер.\,Шогенцукова до МН                          &   3:35    \\
            \hline
        \end{tabular}\hfil}%
    }
    \vspace{0.8cm}

    Утро начинается с прекрасных панорам. Рассветное солнце подсвечивает вершины, а их подножия скрыты в дымке
    облаков. Понимаем, что спускаться нам сегодня в дождливую долину, но пока не спустились, над нами ясное небо.
    В 6:10 собираемся на вершине: подъем проходит по хоженой альпинисткой тропе по осыпи и не представляет сложности.
    Возвращаемся в лагерь и в 7:50 выходим на траверс к перевалу Куллумкол. Траверс к пер. Куллумкол проходит
    по натоптанной альпинистской тропе, но, так как проходили препятствие в начале сезона, в некоторых местах
    нужно быть аккуратными с выходами обледенелых снежников. Встречаются участки простого лазания. На перевале
    собираемся в 9:40.

    Дальше по той же тропе спускаемся с перевала Куллумкол к пер.\,Шогенцукова и далее в долину реки Куллумкол-Су.
    Несмотря на то, что частично путь проходит по ледникам, использование кошек или ледоруба не требуется. На
    перевале Шогенцукова оказываемся в 11:00. Спускаемся с перевала на одни из возможных ночевок (высота 3215\,м),
    обедаем с 12:20 до 13:35.

    На месте впадения реки Куллумкол в Адыр-Су нужно быть аккуратными и идти по маркированной турами тропе!
    Тогда все рукава можно перескочить по камням. Через первый рукав есть надежный мост, далее нужно перейти
    большой рукав по курумнику.

    Выходим на хорошую дорогу, спускаемся к турбазе в 16:35.

    \FloatBarrier
    
    %ЧХВ радиалки: 00:35
    
    %ЧХВ до пер. Куллумкол: 1:25
    %ГХВ до пер. Куллумкол: 1:50
    
    %ЧХВ от пер. Куллумкол до пер. Шогенцукова: 1:05
    %ГХВ от пер. Куллумкол до пер. Шогенцукова: 1:20
    %
    %ЧХВ от пер. Шогенцукова до МН: 3:35
    %ГХВ от пер. Шогенцукова до МН: 4:20
    
    
    %ЧХВ первого:  = ... + 48 + 39 + 60 = ... + 2:27