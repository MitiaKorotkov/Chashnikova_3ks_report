\graphicspath{{Pictures/Chapter5/Day16}}


\subsection{День 16 (15 июля)}\label{subsec:Day16}
    \vbox{%
        \hbox to \textwidth{\hfil%
        \includegraphics[width=0.8\textwidth]{day16.pdf}\label{fig:Day16_map}\hfil}
        \vspace{0.3cm}

        \hbox to \textwidth{\hfil%
        \begin{tabular}{|p{6.1cm}|>{\centering\arraybackslash}p{2.7cm}|}
            \hline
            Расстояние пройденное, км	& 11.2       \\
            Расстояние в зачёт, км      & 9.5  \\
            Из них радиально, км        & 3.5  \\
            Набор высоты, м		& +950/-1500      \\
            Высота ночёвки, м	          & 3023      \\
            \hline
        \end{tabular}\quad%
        \begin{tabular}{|p{6cm}|>{\centering\arraybackslash}p{1.5cm}|}
            \hline
            ЧХВ за день & 10:17 \\
            ГХВ за день  &	12:49	\\			
            \hline
             ЧХВ рад. выхода &  5:25     \\
             ЧХВ от МН до пер. Килар   &	1:37	\\
             ЧХВ от пер. Килар до МН  & 3:15 \\
            \hline
        \end{tabular}\hfil}%
    }
    \vspace{0.8cm}

Сегодня у нас грандиозные планы. Собираемся в радиальный выход: пер. Надежда (1Б, п/п) + пер. Кенчат + вер. Кенчатбаши + пер. Килар (2А). Далее нас ждет перевал Килар насквозь 1Б и подход под ночевки нижние Тютю.

В 3:35 выходим в радиальный выход в сумерках с фонариками. К 4:05 подходим под перевальный взлет перевала Надежда. Склон длиной до 150 м, фирновый, до $35^\circ$. Проходим в кошках, стараемся не идти друг над другом.

В 4:40 собираемся на седловине перевала. Возможно группе с рюкзаками понадобится 1 перильная веревка перед перевальным взлетом.

В 5:30 спустились с перевала: Спуск проходит по средней и мелкой подвижной осыпи крутизной до
$30^\circ$, встречаются выходы конгломерата. В средней части обходим слева скальный выступ, движемся плотной группой.

В 5:40 собираемся у взлёта перевала Кенчат. Подъем на северную седловину перевала Кенчат в начале
по осыпному склону $25^\circ$, затем по смёрзшимся снежникам до $35^\circ$, проходим индивидуально с самостраховкой ледорубом. Собираемся на седловине в 6:25. Седловина перевала широкая, осыпная, с запада прилегает ледник Кенчат Западный, есть небольшое озеро в мульде.

На траверсе к вершине Кенчатбаши движемся в кошках и связках, первый жандарм проходим в
лоб, перелезая через разрушенные скалы и среднюю осыпь, второй жандарм обходим справа
пхд. Перед самой вершиной есть небольшие участки гребневого
лазания. С Кенчатбаши открываются прекрасные виды на массив Дыхтау и Коштантау, вершины
Джайлык и Тютюбаши, Эльбрус, ГКХ. В 7:30 мы на вершине. Спуск по пути подъема. На обходе одного из жандармов от перевала Кенчат к пер. Килар организуем ледобурную дорожку. Скальная
часть гребня вблизи перевала Килар сильно разрушена и требует уверенных навыков гребневого лазания. Спуск проходит по крупной и средней осыпи крутизной до $35^\circ$.

В 10:15 возвращаемся в лагерь.

В 11:10 выходим в направлении перевала Килар. Подъем на перевал через осыпные поля и траверс снежника крутизной до $25^\circ$. В 12:30 собираемся на перевале, просматриваем путь спуска. Видим много расходников, но дюльферять тут совсем не хочется: можно спустить камни друг на друга. В целом склон не очень сложный, можно спуститься плотной группой, обойдя скальные выходы правее пхд.

Спускаемся траверсом крупной осыпи с выходами разрушенных скал, встречаются элементы лазания. Крутизна до $35^\circ$. В 14:05 собираемся на леднике.

В связкам проходим ледник, зону трещин обходим справа пхд. Выходим на осыпь, делаем обед в 15:20.
Далее наш путь пролегает по моренной помоечке, на разливе переходим на орографически левый берег и движемся по нему. Такой выбор траектории правильный: бродить реку у места ночевки было бы непросто. Разливы реки из-под ледника Тютю не обозначены на карте, но видны на спутниковом снимке.

В 18:55 приходим на место ночевки: здесь много хороших площадок под палатки, но нет чистой воды, пьем мутную из речки.

%подъём 2:30

%~3:35 выход на радиалку (Настя Лёша я Катя)

%4:05 мы у пер взлёта, надеваем кошки

%4:13 -- 4:40 забежали в кошках на седловину; с рюкзаками в верхней части взлёта нужна будет 1 верёвка по снегу / льду.
%крутизна этой верёвки 35-45 градусов. у нас фирн

%4:55 -- 5:27 спускаемся до низу почти
%сначала средняя осыпь живая, по ней плотной группой в средней части обходим слева скальный выступ, далее направол к кенчату

%5:38 у взлёта пер кенчату

%5:45 -- 6:25 (6:30) забежали в кошках на пер кенчат в верх части подъёма идём по снежнику 45

%646 -- 740
%745 -- 845
%900 -- 1015

%... gps?

%8:50 -- ??? вышли на осыпной гребень, вед к пер Килар
%Мы обходили по снегу жанд по ледобурной дорожке из 2 буров

%10:16 мы в лагере

%11:13 -- 11:49 (11:53) идём по леднику к перевалу Килар h=3790

%12:03 -- 12:30 (13:00) h=3915 идём по осыпи к перевалу, далее выходим на снежник и траверсим его по
%осыпи / скалам на пер.

%13:15 -- 14:05 спуск по живой крупной осыпи до ледника
%В нижней части уходим вправо, перевал говна

%14:35 -- 15:17 прошли в связках пер
%Разломы справа легко обходятся (слева пхд), верёвки тут для лалак

%обед h~3500

%16:38 -- 17:19 (17:22 посл)
%спускаемся по левому берегу реки
%В крутой части спуска выходим на конгломератный гребень
%Прошли в начале ходки первые возможные ночёвки
%h=3185

%17:32(?) -- 19:00 h=3030 пришли на ночёвку, шли по левому берегу реки, стоило идти по правому (по треку)
%на левом берегу реки -- моренные валы, +сыпуха на льду вода из речки, можно фильтровать или отстаивать

%готово
%ЧХВ радиалки: 5:25
%ГХВ радиалки: 6:40

%чхв 1го от МН до пер Килар: 1:03
%чхв группы от МН до пер Килар: 1:37
%гхв группы от МН до пер Килар: 1:50

%чхв 1го от пер Килар до ночёвки
%чхв группы от пер Килар до ночёвки 3:15
%гхв группы от пер Килар до ночёвки 259 = 4:19

%чхв за день: 10:17
%гхв за день: 12:49

%ЧХВ группы: 40 + 57 + 50 + 42 + 44 + 88 = 5:21
%ЧХВ первого: 36 + 27 + 50 + 42 + 41 + 88 = 4:44
%Гхв группы:
    \FloatBarrier
