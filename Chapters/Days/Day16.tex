\graphicspath{{Pictures/Chapter5/Day16}}


\subsection{День 16 (15 июля)}\label{subsec:Day16}
    \vbox{%
        \hbox to \textwidth{\hfil%
        \includegraphics[width=0.8\textwidth]{day16.pdf}\label{fig:Day16_map}\hfil}
        \vspace{0.3cm}

        \hbox to \textwidth{\hfil%
        \begin{tabular}{|p{4.5cm}|>{\centering\arraybackslash}p{4cm}|}
            \hline
            Расстояние, км		&       \\
            Набор высоты, м		&       \\
            Высота ночёвки, м	&       \\
            Метеоусловия		&       \\
            Покрытие			&       \\
            \hline
        \end{tabular}\quad%
        \begin{tabular}{|p{5cm}|>{\centering\arraybackslash}p{1.5cm}|}
            \hline
                &		\\			
            \hline
                &       \\
                &		\\
            \hline
        \end{tabular}\hfil}%
    }
    \vspace{0.8cm}
    
подъём 2:30

~3:35 выход на радиалку (Настя Лёша я Катя)

4:05 мы у пер взлёта, надеваем кошки

4:13 -- 4:40 забежали в кошках на седловину; с рюкзаками в верхней части взлёта нужна будет 1 верёвка по снегу / льду.
крутизна этой верёвки 35-45 градусов. у нас фирн

4:55 -- 5:27 спускаемся до низу почти
сначала средняя осыпь живая, по ней плотной группой в средней части обходим слева скальный выступ, далее направол к кенчату

5:38 у взлёта пер кенчату

5:45 -- 6:25 (6:30) забежали в кошках на пер кенчат в верх части подъёма идём по снежнику 45

... gps?

8:50 -- ??? вышли на осыпной гребень, вед к пер Килар
Мы обходили по снегу жанд по ледобурной дорожке из 2 буров

10:16 мы в лагере

11:13 -- 11:49 (11:53) идём по леднику к перевалу Килар h=3790

12:03 -- 12:30 (13:00) h=3915 идём по осыпи к перевалу, далее выходим на снежник и траверсим его по
осыпи / скалам на пер.

13:15 -- 14:05 спуск по живой крупной осыпи до ледника
В нижней части уходим вправо, перевал говна

14:35 -- 15:17 прошли в связках пер
Разломы справа легко обходятся (слева пхд), верёвки тут для лалак

обед h~3500

16:38 -- 17:19 (17:22 посл)
спускаемся по левому берегу реки
В крутой части спуска выходим на конгломератный гребень
Прошли в начале ходки первые возможные ночёвки
h=3185

17:32(?) -- 19:00 h=3030 пришли на ночёвку, шли по левому берегу реки, стоило идти по правому (по треку)
на левом берегу реки -- моренные валы, +сыпуха на льду вода из речки, можно фильтровать или отстаивать


ЧХВ радиалки: 30 + 27 + 43 + 45 + ??? = 2:25 + ???

ЧХВ группы: 40 + 57 + 50 + 42 + 44 + 88 = 5:21

ЧХВ первого: 36 + 27 + 50 + 42 + 41 + 88 = 4:44

    \FloatBarrier
