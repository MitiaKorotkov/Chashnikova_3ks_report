\graphicspath{{Pictures/Chapter5/Day1/}}

\subsection{День 1 (30 июня)}\label{subsec:Day1}
		
    \begin{figure}[ht]
        \centering
        \includegraphics[width=0.6\textwidth, angle=90]{Pictures/Chapter5/Day1/day1.pdf}\label{fig:Day1_map}

        
        \begin{tabular}{|p{4.5cm}|>{\centering\arraybackslash}p{4cm}|}
            \hline
            Расстояние, км		&	1.6								\\
            Набор высоты, м		&	+10								\\
            Высота ночёвки, м	&	1520							\\
            Метеоусловия		&	Солнечно, небольшая облачность	\\
            Покрытие			&	Грунтовая дорога				\\
            \hline
        \end{tabular}\quad
        \begin{tabular}{|p{5cm}|>{\centering\arraybackslash}p{1.5cm}|}
            \hline
            ЧХВ от аула Верхний Учкулан до МН	&	0:20	\\			
            \hline
            Итого ЧХВ							&	0:20	\\
            Итого ГХВ							&	0:20	\\
            \hline
        \end{tabular}
    \end{figure}

   % Трек похода на \href{https://nakarte.me/#m=14/43.43375/42.03412&l=Otm/Wp&nktl=BdWP0CRtjQw-_dLu4ICnfQ}{карте}
    
    С утра, в 3:40, выгрузившись из поезда на вокзале Минеральных Вод, ждём двух участниц, которые должны
    приехать из Пятигорска и фасуем продукты. В 11:15 за нами приезжает автобус. Погружаемся и, заехав
    в аэропорт за ещё одной участницей, прилетевшей самолётом, направляемся в Верхний Учкулан. По пути
    отдаём местным ребятам заброску и в 17:00 мы на месте. Выгружаемся из автобуса,
    фотографируемся\prettyref{fig:Day1_2} и в 17:34 начали движжение. Через 20 минут нашли подходящее
    место для ночёвки и встали лагерем на берегу реки Узункол.

1) 7:27 -- 8:13  (посл 8:15) (1800 -> 1915 м) На выходе с базы слушаем инструктаж и смотрим на убитого волка. Спуск до моста через 400м по горизонтали, далее набор по автомобильной дороге по правому орогр. берегу реки. Дошли до верхней турбазы

2) 8:25 -- 9:10 (посл 9:22)  (1915 -> 2285 м ?? уточнить высоту в гпс) Идём по правому орогр берегу реки по автом дороге. На другом берегу -- погранзастава с табличкой

3) 9:32 -- 10:06 (посл 10:12) (?? -> 2300 м) "шоссе" кончилось, далее проезжая дорога по долине. в начале ходки проходим питьевой ручей с водопадом

4) 10:23 -- 11:08 (посл 11:12) (... -> 2450 м) около 10 мин бродим приток, я и Лёша по камням, девочки переобулись. Ниже по течению есть мост. Дорога стала тропой.

5) 11:22 -- 12:08 (посл 12:10) (... -> 2550 м) перешли реку, +- прыгается по камням, девочки переобулись. Вышли на морену, заросшую травой, в середине ходки

6) 12:10 -- 12:43 (посл 12:50) (... -> 2650 м) подошли к озеру, чистим от камней берег. Обошли справа пхд моренную помойку, стоим у истока ручья. Обед (до 13:46).

чхв 1-го за день: 46 + 45 + 34 + 45 + 46 + 33 = 249 мин = 04:09

чхв группы за день: 48 + 57 + 40 + 49 + 48 + 40 = 04:42

    \begin{figure}[hb]
        \centering
        \includegraphics[width=\textwidth]{Pictures/Chapter5/Day1/2.jpeg}
        \caption{Группа выгрузилась из автобуса}
        \label{fig:Day1_2}
    \end{figure}

    \FloatBarrier