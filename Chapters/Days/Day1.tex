\graphicspath{{Pictures/Chapter5/Day1/}}

\subsection{День 1 (30 июня)}\label{subsec:Day1}
		
    \vbox{%
        \hbox to \textwidth{%
        \hfil\includegraphics[width=0.6\textwidth, angle=90]{day1sm.pdf}\label{fig:Day1_map}\hfil}
        \vspace{0.3cm}

        \hbox to \textwidth{\hfil%
        \begin{tabular}{|p{4.5cm}|>{\centering\arraybackslash}p{2.7cm}|}
            \hline
            Расстояние, км		&	12      \\
            Набор высоты, м		&	+850	\\
            Высота ночёвки, м	&	2663	\\
            \hline
        \end{tabular}\quad%
        \begin{tabular}{|p{5cm}|>{\centering\arraybackslash}p{1.5cm}|}
            \hline
            ЧХВ за день &	04:42   \\
            ГХВ за день	&	05:23	\\
            \hline
        \end{tabular}\hfil}%
    }
    \vspace{0.8cm}

    Ночуем на нижней турбазе Уштулу (на высоте 1850\,м). Выходим с турбазы в 7:25. Спускаемся от турбазы к развилке,
    далее плавно набираем высоту по автомобильной дороге по орографически правому берегу реки. Верхняя турбаза в одной
    ходке. В 10:15 Автомобильная дорога заканчивается (высота 2250\,м, прошли около 6\,км). Продолжаем движение по
    проселочной дороге по долине. В начале ходки проходим питьевой ручей с водопадом. Приток Ахсу легко перейти,
    переобувшись в бродную обувь. После брода замечаем мост ниже по течению. Дорога становится не такой явной и
    превращается в тропу. В 11:30 подходим к разливам р. Карасу у морены ледника Штулу. Течение не сильное, глубина
    ниже колена девочкам. После брода разливов Карасу выходим на морену, заросшую травой. Мы идем по гребню моренного
    холма, правее пхд идет тропа траверсом травянистого склона. В 12:50 дошли до озера в кармане морены на высоте
    2650\,м. Встали на ночевку у северного берега озера, воду берем из ручья, вытекающего из озера.

    \FloatBarrier
