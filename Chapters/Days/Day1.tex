\graphicspath{{Pictures/Chapter5/Day1/}}

\subsection{День 1 (30 июня)}\label{subsec:Day1}
		
    \vbox{%
        \hbox to \textwidth{%
        \hfil\includegraphics[width=0.6\textwidth, angle=90]{day1.pdf}\label{fig:Day1_map}\hfil}
        \vspace{0.3cm}

        \hbox to \textwidth{\hfil%
        \begin{tabular}{|p{4.5cm}|>{\centering\arraybackslash}p{4cm}|}
            \hline
            Расстояние, км		&	12								\\
            Набор высоты, м		&	+850						\\
            Высота ночёвки, м	&	2650							\\
   %         Метеоусловия		&		                            \\
   %         Покрытие			&	Грунтовая дорога				\\
            \hline
        \end{tabular}\quad%
        \begin{tabular}{|p{5cm}|>{\centering\arraybackslash}p{1.5cm}|}
            \hline
            ЧХВ группы за день							&	04:42	\\
            ГХВ группы за день							&	05:23	\\
            \hline
        \end{tabular}\hfil}%
    }
    \vspace{0.8cm}

  % Трек похода на \href{https://nakarte.me/#m=14/43.43375/42.03412&l=Otm/Wp&nktl=BdWP0CRtjQw-_dLu4ICnfQ}{карте}

Ночуем на нижней турбазе Уштулу (на высоте 1850 м). Выходим с турбазы в 7:25. Спускаемся от турбазы к развилке, далее плавно набираем высоту по автомобильной дороге по орографически правому берегу реки. Верхняя турбаза в одной ходке. В 10:15 Автомобильная дорога заканчивается (высота 2250 м, прошли около 6 км). Продолжаем движение по проселочной дороге по долине. В начале ходки проходим питьевой ручей с водопадом. Приток Ахсу легко перейти, переобувшись в бродную обувь. После брода замечаем мост ниже по течению. Дорога становится не такой явно и превращается в тропу. В 11-30 подходим к разливам Карасу у морены. Течение не сильное, глубина ниже колена девочкам. После брода разливов Карасу выходим на морену, заросшую травой. Мы идем по гребню моренного холма, правее пхд идет тропа траверсом травянистого склона. В 12:50 дошли до озера в кармане морены на высоте 2650 м. Встали на ночевку у северного берега озера, воду берем из ручья, вытекающего из озера.

%ЧХВ группы за день 04:42


%далее оригинал хронометража

%1) 7:27 -- 8:13  (посл 8:15) (1800 -> 1915 м) На выходе с базы слушаем инструктаж и смотрим на убитого волка. Спуск до моста через 400м по горизонтали, далее набор по автомобильной дороге по правому орогр. берегу реки. Дошли до верхней турбазы

%2) 8:25 -- 9:10 (посл 9:22)  (1915 -> 2285 м ?? уточнить высоту в гпс) Идём по правому орогр берегу реки по автом дороге. На другом берегу -- погранзастава с табличкой

%3) 9:32 -- 10:06 (посл 10:12) (?? -> 2300 м) "шоссе" кончилось, далее проезжая дорога по долине. в начале ходки проходим питьевой ручей с водопадом

%4) 10:23 -- 11:08 (посл 11:12) (... -> 2450 м) около 10 мин бродим приток, я и Лёша по камням, девочки переобулись. Ниже по течению есть мост. Дорога стала тропой.

%5) 11:22 -- 12:08 (посл 12:10) (... -> 2550 м) перешли реку, +- прыгается по камням, девочки переобулись. Вышли на морену, заросшую травой, в середине ходки

%6) 12:10 -- 12:43 (посл 12:50) (... -> 2650 м) подошли к озеру, чистим от камней берег. Обошли справа пхд моренную помойку, стоим у истока ручья. Обед (до 13:46).

%чхв 1-го за день: 46 + 45 + 34 + 45 + 46 + 33 = 249 мин = 04:09

%чхв группы за день: 48 + 57 + 40 + 49 + 48 + 40 = 04:42

    \FloatBarrier
