\graphicspath{Pictures/Chapter5/Day5}


\subsection{День 5 (4 июля)}\label{subsec:Day5}
    \begin{figure}[ht]
        \centering
        \includegraphics[width=0.6\textwidth, angle=90]{Pictures/Chapter5/Day5/day5.pdf}\label{fig:Day5_map}

        \begin{tabular}{|p{4.5cm}|>{\centering\arraybackslash}p{4cm}|}
            \hline
            Расстояние, км		&       \\
            Набор высоты, м		&       \\
            Высота ночёвки, м	&       \\
            Метеоусловия		&       \\
            Покрытие			&       \\
            \hline
        \end{tabular}\quad
        \begin{tabular}{|p{5cm}|>{\centering\arraybackslash}p{1.5cm}|}
            \hline
            	&		\\			
            \hline
                &       \\
                &		\\
            \hline
        \end{tabular}
    \end{figure}

1) 6:27 -- 7:07 (посл 7:17) Направо по морене пхд к след перевалу. Перед выходом на снег надеваем кошки и обвязки h=3575

2) 7:28 -- 8:14 (посл 8:21) h=3735 (выход по готовности) Затропили на перевал, ясная погода. Нужная седловина -0- правая пхд. Скалы и участки осыпи обходим слева. Можно было прямее. Выход в 855. На перевале два тура, записка была в правом. Навстречу поднимается группа

3) 8:55 -- 9:50 (?) Сначала лифт до снежника. Снежник можно идти в узкой части, можно весь. Далее через ~100м снова снежник. После выхрнего снежника немного лифта вдоль след снежника. Лёша бегает на разведку.

4) 10:? -- 11:00 (посл 11:03) h=3235 Спускаем по осыпи и снежникам ло зелёного гребня. На наиболее крутом снежнике челночим рюкзаки девушек (ступени, твёрдый снег, 3 такта).

5) 11:25...11:32 -- 12:12 (посл 12:22) Идём по морене (сначала по гребню зелёному)

6) 12:32 -- 13:16 (посл 13:18) h=3350 Прошкрябались по полуоткрытому леднику до морены. В 1330 начали вязать связки. Обед (зачёркнуто)
Прошли ещё немного, Настя, Лёша и Дима ьидут на разведку. Обед (и ночёвка?) Сеансы связи: 1600, 1630, 1700 и т.д.

Далее видимо была трена или что?


    \FloatBarrier