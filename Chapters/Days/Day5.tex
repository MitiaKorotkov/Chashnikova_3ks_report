\graphicspath{{Pictures/Chapter5/Day5}}


\subsection{День 5 (4 июля)}\label{subsec:Day5}
    \vbox{%
        \hbox to \textwidth{\hfil%
        \includegraphics[width=0.9\textwidth]{day5sm.pdf}\label{fig:Day5_map}\hfil}
        \vspace{0.3cm}

        \hbox to \textwidth{\hfil%
        \begin{tabular}{|p{4.5cm}|>{\centering\arraybackslash}p{4cm}|}
            \hline
            Расстояние, км		&   6.0    \\
            Набор высоты, м		&   +550/-500    \\
            Высота ночёвки, м	&    3453   \\
            \hline
        \end{tabular}\quad%
        \begin{tabular}{|p{5cm}|>{\centering\arraybackslash}p{1.5cm}|}
            \hline
                ЧХВ за день &	6:10	\\		
                ГХВ за день &	7:45	\\		
            \hline
                ЧХВ от МН до пер. Ашинова & 1:50  \\
                ЧХВ от пер. Ашинова до МН & 4:40	\\
            \hline
        \end{tabular}\hfil}%
    }
    \vspace{0.8cm}


Выходим с места ночевки в 6:30. Движемся по осыпным полям к перевалу, участки осыпи перемежаются со снежниками.

Перевальный взлет снежный, смёрзшийся, до $30^\circ$ крутизной. Перед выходом на снег надеваем кошки, делаем ступени. Поднимаемся по одному с самостраховкой ледорубом, стараемся не ходить друг над другом. После подъёма траверсируем гребень вправо пхд по пологому снежнику. В 8:20 собираемся на перевале. Седловина широкая, осыпная.

В 8:55 начинаем спуск с перевала, сначала по лифтовой осыпи, затем траверсируем смерзшийся снежник в кошках (до $20^\circ$ крутизной). Спускаемся по осыпи и снежникам до заросшего гребня морены (11:15). Борта морены постепенно становятся менее крутыми, просматривается выход на ледник Крумкол.

В 13:15 выходим на ледник Крумкол, на его срединную морену (высота 3350 м). Далее ледник закрывается, идем в связках. В 13:20 начали вязать связки, к 14:50 подходим на хорошее место ночевки. Три участника идут на разведку ледопада, который нам предстоит пройти завтра.

%1) 6:27 -- 7:07 (посл 7:17) Направо по морене пхд к след перевалу. Перед выходом на снег надеваем кошки и обвязки h=3575

%2) 7:28 -- 8:14 (посл 8:21) h=3735 (выход по готовности) Затропили на перевал, ясная погода. Нужная седловина -- правая пхд. Скалы и участки осыпи обходим слева. Можно было прямее. На перевале два тура, записка была в правом. Навстречу поднимается группа

%3) 8:55 -- 9:50 (?) Сначала лифт до снежника. Снежник можно идти в узкой части, можно весь. Далее через ~100м снова снежник. После верхнего снежника немного лифта вдоль след снежника. Лёша бегает на разведку.

%4) 10:? -- 11:00 (посл 11:03) h=3235 Спускаем по осыпи и снежникам ло зелёного гребня. На наиболее крутом снежнике челночим рюкзаки девушек (ступени, твёрдый снег, 3 такта).

%5) 11:25...11:32 -- 12:12 (посл 12:22) Идём по морене (сначала по гребню зелёному)

%6) 12:32 -- 13:16 (посл 13:18) h=3350 Прошкрябались по полуоткрытому леднику до морены. В 1330 начали вязать связки. Обед (зачёркнуто)
%Прошли ещё немного, Настя, Лёша и Дима ьидут на разведку. Обед (и ночёвка?) Сеансы связи: 1600, 1630, 1700 и т.д.

%готово
%чхв 1го до перевала 86 мин = 1:26
%чхв группы до перевала ~1:50
%чхв группы до перевала 1:30 %примерно
%гхв группы до перевала 1:50

%чхв группы от перевала до МН: 4:40 %примерно
%гхв группы от перевала до МН: 5:55

    \FloatBarrier
