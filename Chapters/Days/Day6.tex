\graphicspath{{Pictures/Chapter5/Day6}}


\subsection{День 6 (5 июля)}\label{subsec:Day6}
    \vbox{%
        \hbox to \textwidth{\hfil%
        \includegraphics[width=0.9\textwidth]{day6sm.pdf}\label{fig:Day6_map}\hfil}
        \vspace{0.3cm}

        \hbox to \textwidth{\hfil%
        \begin{tabular}{|p{4.5cm}|>{\centering\arraybackslash}p{4cm}|}
            \hline
            Расстояние, км		&   4.0    \\
            Набор высоты, м		&   +650    \\
            Высота ночёвки, м	&    4091   \\
            \hline
        \end{tabular}\quad%
        \begin{tabular}{|p{5cm}|>{\centering\arraybackslash}p{1.5cm}|}
            \hline
                ЧХВ за день &  4:32     \\
                ГХВ за день &	5:30	\\
            \hline
        \end{tabular}\hfil}%
    }
    \vspace{0.8cm}

Сегодня проснулись пораньше, чтобы пройти ледопад рано утром и перейти трещины по надежным мостам. Выходим по готовности, первая связка-тройка в 5:05, вторая связка-четвёрка в 5:20. В 6:15 подходим к наиболее крутой и разорванной части ледопада, собираемся вместе. Погода  ясная, снег твёрдый, трещины забиты очень плотным и надежным снегом. Почти везде можно пройти с одновременной страховкой в связках, в нескольких наиболее крутых участках (до $40^\circ$ крутизной) делаем ледобурную дорожку с попеременной страховкой, проходим на передних зубьях.
Заканчиваем прохождение ледопада в 8:40.

Далее ледник пологий и спокойный. Перед перевалом огромное плато, взлет на перевал - короткий снежный склон до $25^\circ$ крутизной.

Собираемся на перевале в 10:25. Седловина перевала неширокая, достаточно долго ровняем площадки, чтобы поставить на них палатки. Открывается прекрасный вид на Айламу и Шхару. Отдыхаем.

%-------------------------------------
%НЕ УДАЛЯТЬ!
%готово
%ходки трека:

%5:05 -- 5:46

%6:00 -- 6:25

%далее делаем страховку несколько раз?
%видимо с 6:25 тут было прохождение сложного участка со страховкой

%7:00 -- ? 7:27

%7:34 -- 7:45

%8:00 -- 8:06

%8:21 -- 8:40

%далее пошли нормальные ходки:

%8:58 -- 9:34

%9:40 -- 9:55 (первого!)

%чхв 1го 217 мин = 3:37
%чхв группы 272 мин = 4:32 %неточно!
%гхв группы 5:25

%Подъём 3
%выход 1ой связки в 505, 2ой связки в 521, к 6:14 почти догнали первую связку
%Делаем привал перед крутым участком перед ледопадом
%Погода ясная, снег твёрдый. Ледопад скоро будет под солнцем

%6:53 встретились с 1ой связкой

%1ая выходит в 6:58
%1ая связка залезла на пер в 955

%собрались на перевале в 1028

%При прохождении перевала мы прошли зону разломов в 840
%Далее 2 связки идут в своём темпе, иногда делая привалы по необходимости


    \FloatBarrier
