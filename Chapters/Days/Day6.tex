\graphicspath{Pictures/Chapter5/Day6}


\subsection{День 6 (5 июля)}\label{subsec:Day6}
    \begin{figure}[ht]
        \centering
        \includegraphics[width=0.9\textwidth]{Pictures/Chapter5/Day6/day6.pdf}\label{fig:Day6_map}

        \begin{tabular}{|p{4.5cm}|>{\centering\arraybackslash}p{4cm}|}
            \hline
            Расстояние, км		&   3.9    \\
            Набор высоты, м		&   +642    \\
            Высота ночёвки, м	&    4116   \\
            Метеоусловия		&       \\
            Покрытие			&       \\
            \hline
        \end{tabular}\quad
        \begin{tabular}{|p{5cm}|>{\centering\arraybackslash}p{1.5cm}|}
            \hline
            	&		\\			
            \hline
                &       \\
                &		\\
            \hline
        \end{tabular}
    \end{figure}

Подъём 3
выход 1ой связки в 505, 2ой связки в 521, к 6:14 почти догнали первую связку
Делаем привал перед крутым участком перед ледопадом
Погода ясная, снег твёрдый. Ледопад скоро будет под солнцем

6:53 встретились с 1ой связкой

1ая выходит в 6:58
1ая связка залезла на пер в 955

собрались на перевале в 1028

При прохождении перевала мы прошли зону разломов в 840
Далее 2 связки идут в своём темпе, иногда делая привалы по необходимости


    \FloatBarrier
