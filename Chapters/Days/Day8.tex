\graphicspath{{Pictures/Chapter5/Day8}}


\subsection{День 8 (7 июля)}\label{subsec:Day8}
    \vbox{%
        \hbox to \textwidth{\hfil%
        \includegraphics[width=0.7\textwidth]{day8.pdf}\label{fig:Day8_map}\hfil}
        \vspace{0.3cm}

        \hbox to \textwidth{\hfil%
        \begin{tabular}{|p{4.5cm}|>{\centering\arraybackslash}p{4cm}|}
            \hline
            Расстояние, км		&       \\
            Набор высоты, м		&       \\
            Высота ночёвки, м	&       \\
            Метеоусловия		&       \\
            Покрытие			&       \\
            \hline
        \end{tabular}\quad%
        \begin{tabular}{|p{5cm}|>{\centering\arraybackslash}p{1.5cm}|}
            \hline
                &		\\			
            \hline
                &       \\
                &		\\
            \hline
        \end{tabular}\hfil}%
    }
    \vspace{0.8cm}
    
поздний подъём, устали
1) 915 -- 1018 (посл 1023) обходим трещины, долго h=3900

2) 1034 -- 1120 доходим до ледопада, участок зачехлённого камнями открытого ледника идём без кошек. h=3700

3) 1135 -- 1212 (посл 1216) h=3320 идём по правому борту Безенгийского ледника. связки развязали в конце ходки. В августе тут осыпь, у нас осыпь + снежники

4) 1230 -- 13 ходка тропа около Джанги-Коша маркирована турами. После Д-К тропа ведёт на правую морену Без. ледника. Тратим время на снятие кошек, разутепление.

5) 1308 -- 1404 (посл 1406) идём дальше по тропе вдоль ледника

6) 1416 -- 1434 дошли по тропе до наблюдательного пункта пограничников, обед. В обед пережидаем дождь.

7) 1722 -- 1908 тропа идёт от угловых ночёвок вниз через водопад к леднику. В некоторых местах тропа размыта. Выходим на центральное шоссе ледника, он идётся без кошек
Подходим к Баран-кошу

8) 1912 -- 1938 дошли до стоянок на баран-коше, ночёвка, ура! h=2760

    \FloatBarrier
