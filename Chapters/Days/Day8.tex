\graphicspath{{Pictures/Chapter5/Day8}}


\subsection{День 8 (7 июля)}\label{subsec:Day8}
    \vbox{%
        \hbox to \textwidth{\hfil%
        \includegraphics[width=0.7\textwidth]{day8.pdf}\label{fig:Day8_map}\hfil}
        \vspace{0.3cm}

        \hbox to \textwidth{\hfil%
        \begin{tabular}{|p{4.5cm}|>{\centering\arraybackslash}p{3cm}|}
            \hline
            Расстояние, км		&   11.6    \\
            Набор высоты, м		&   -1500    \\
            Высота ночёвки, м	&      2755 \\
             \hline
        \end{tabular}\quad%
        \begin{tabular}{|p{6cm}|>{\centering\arraybackslash}p{1.5cm}|}
            \hline
                ЧХВ за день &	6:35	\\		
                ГХВ за день &	7:44	\\		
            \hline
                ЧХВ от МН до м. обеда &   4:19    \\
                ЧХВ от м. обеда до МН &	 2:16	\\
            \hline
        \end{tabular}\hfil}%
    }
    \vspace{0.8cm}

Просыпаемся -- вокруг пурга, видимость очень низкая. Принимаем решение поспать еще и подождать, пока погода улучшится. В 7:30 все так же идет снег, но видимость хорошая. Собираем лагерь, в 9:15 выходим в связках вниз по леднику Уллучиран. Снегопад усилился, видимость метров 200 -- 300. Трещины видно, их достаточно много, большинство легко перепрыгнуть или найти путь обхода. Штурман идет очень аккуратно, в трещины больше никто не провалился. В 11:20 доходим до зачехлённого камнями открытого участка ледника, развязываем связки. По снежнику выходим на морену с правого борта ледника Уллучиран, идем по чьим-то следам и выходим на тропу. В 12:30 подходим к Джанги-Кошу, здесь прекрасное место для ночевки. К 14:35 дошли по тропе до угловых ночевок. Начинается дождь, растягиваем тент и обедаем. Воду можно набрать из ручья. В 17:20 выходим с обеда, дождь прекратился. Наш план на сегодня -- дойти до Баранкоша.

Тропа идёт от угловых ночёвок вниз через водопад к леднику. В некоторых местах тропа размыта и обрушена. Через крутой конгломератный склон выходим с морены на ледник, рядом с мореной много трещин, аккуратно обходим их и выходим на центральную часть ледника. Тут плоско и нет трещин. В 19:40 приходим на озеро на место ночевки. Погода снова улучшилась и наконец-то открылась Безенгийская стена.
Важная информация: Специалистами ФГБУ «Высокогорного Геофизического института» в результате анализа разновременных космоснимков было выявлено, что в начале мая 2023 года на участке Безенгийской стены под горой Джангитау Западная (5059 м) образовалась трещина, которая непрерывно увеличивается. Не рекомендовано останавливаться на ночлег на Судейских ночевках Северного массива и Угловых (Чешских) ночевках Безенгийского ущелья.

%поздний подъём, устали
%1) 915 -- 1018 (посл 1023) обходим трещины, долго h=3900

%2) 1034 -- 1120 доходим до ледопада, участок зачехлённого камнями открытого ледника идём без кошек. h=3700

%3) 1135 -- 1212 (посл 1216) h=3320 идём по правому борту Безенгийского ледника. связки развязали в конце ходки. В августе тут осыпь, у нас осыпь + снежники

%4) 1230 -- 13 ходка тропа около Джанги-Коша маркирована турами. После Д-К тропа ведёт на правую морену Без. ледника. Тратим время на снятие кошек, разутепление.

%5) 1308 -- 1404 (посл 1406) идём дальше по тропе вдоль ледника

%6) 1416 -- 1434 дошли по тропе до наблюдательного пункта пограничников, обед. В обед пережидаем дождь.

%7) 1722 -- 1908? тропа идёт от угловых ночёвок вниз через водопад к леднику. В некоторых местах тропа размыта. Выходим на центральное шоссе ледника, он идётся без кошек
%Подходим к Баран-кошу

%8) 1912 -- 1938 дошли до стоянок на баран-коше, ночёвка, ура! h=2760

%готово
%чхв 1го до обеда 4:08
%чхв группы до обеда 4:19
%гхв группы до обеда 5:19

%чхв 1го от обеда до ночёвки 2:12
%чхв группы от обеда до ночёвки 2:16
%гхв группы от обеда до ночёвки 2:25 (неточно)

%чхв 1го за день 6:20
%чхв группы за день 6:35
%гхв группы за день 7:44

    \FloatBarrier
