\graphicspath{{Pictures/Chapter5/Day17}}

\subsection{День 17 (16 июля)}\label{subsec:Day17}

    \parbox[c]{\textwidth}{%
        \includegraphics[width=0.45\textwidth]{day17sm.pdf}\label{fig:Day17_map}%
        \raisebox{0.6\height}{\vbox{%
            \hbox to 0.5\textwidth{\hfil%
                \begin{tabular}{|p{4.5cm}|>{\centering\arraybackslash}p{2.2cm}|}
                    \hline
                    Расстояние, км	    &   $3,5$   \\
                    Набор высоты, м	    &   $+1180$ \\
                    Высота ночёвки, м	&   $4186$  \\
                    \hline
                \end{tabular}\hfil}%
            \vspace{0.3cm}

            \hbox to 0.5\textwidth{\hfil%
                \begin{tabular}{|p{3cm}|>{\centering\arraybackslash}p{1.5cm}|}
                    \hline
                    ЧХВ за день   & 5:35    \\
                    ГХВ за день   & 8:20	\\
                    \hline
                \end{tabular}\hfil}%
        }}%
    }
    \vspace{0.8cm}
    \fxnote{Сделать КАРТИНКУ В ТЕКСТЕ! то есть текст должен быть справа от карты в том числе}

    В 5:10 выходим. Снова обещали плохую погоду после обеда, а проходить зону трещин в пургу не хочется. До ледника
    Тютю движемся по тропе, маркированной турами. За ходку доходим до ночевок Тютю верхние, встречаем группу.
    Набираем чистую воду. В 6:15 дошли до ледника, связываемся, надеваем кошки. Подъем по леднику можно
    охарактеризовать так: крутой участок (до $25^\circ$), потом пологий участок, потом короткий крутой участок
    (до $25^\circ$), потом снова пологий участок, потом очень крутой участок с трещинами (до $40^\circ$), потом
    короткий пологий участок с трещинами, потом бергшрунд большой, потом пологий участок, потом перевальный взлет
    (до $25^\circ$). Возможны ночевки на скальном отрожке на высоте 3850\,м.

    Так как середина июля "--- многие трещины забиты снегом. Трещины несложно перейти по мостам. Ледовые участки на
    траверсе до $30^\circ$ проходим с ледобурной дорожкой. Наиболее крутой участок (до $40^\circ$) проходим с
    перильной страховкой. Перед выходом в перевальный цирк переходим крупный бергшрунд по надежному снежному мосту.

    Перевальный взлет короткий, снежный, крутизной до $25^\circ$. На гребне группа собирается в 13:10. В мульде
    озеро с чистой водой. Обедаем и идем западнее, на перевал Тютю Западный, ищем хорошие места под палатки, ровняем.
    Здесь отличная обзорная точка. Лучшие утренние панорамы сделаны именно отсюда, крайне рекомендуем. В 15:00
    поставили палатки, пошёл снег.

    \FloatBarrier
    
    %готово? неточно?
    
    
    %подъём 330
    
    %5:13 --  5:54 (6:02) h=3250
    %набираем по тропе к леднику Тютю(?) На тропе есть пара туров. На месте привала есть чистая вода.
    
    %6:13 -- 6:23 дошли до ледника, связываемся
    
    %6:52 -- 7:33 (7:38) h=3510 набираем в связках по леднику
    
    %7:50 -- 8:29 (8:34) h=3710
    
    %8:45 -- 9:21 (9:26) h=3870
    %ещё ступень ледника; возможны ночёвки на скальном отрожке ~3850
    
    %? -- 11:20 (11:37 1ая св) ледорубная дорожка h=3980 связка по связке -> ледобурн дорожка
    %забрались на +1 ступень
    
    %11:40 (11:55 1ая связка) -- 12:40 (12:48 2ая связка)
    %почти дошли в связках на перевал. проходим берг со страховкой
    
    %13:00 -- 13:15 (13:20) вылезли на не перевал не стоянку но обед
    
    %+10 мин чхв до ночёвки
    
    %ЧХВ 1го: 41 + 10 + 41 + 39 + 36 + ??? + 45 + 15 + 10 = ??? + 3:57
    %ЧХВ группы: 49 + 10 + 46 + 44 + 41 + ??? + 68 + 20 + 10 = ??? + 4:48
