\graphicspath{{Pictures/Chapter5/Day17}}

\fxnote{Сделать КАРТИНКУ В ТЕКСТЕ! то есть текст должен быть справа от карты в том числе}

\subsection{День 17 (16 июля)}\label{subsec:Day17}
    \parbox[c]{\textwidth}{%
        \includegraphics[width=0.45\textwidth]{day17.pdf}\label{fig:Day17_map}%
        \raisebox{0.6\height}{\vbox{%
            \hbox to 0.53\textwidth{\hfil%
                \begin{tabular}{|p{4.5cm}|>{\centering\arraybackslash}p{4cm}|}
                    \hline
                    Расстояние, км		&    \\
                    Набор высоты, м		&   \\
                    Высота ночёвки, м	&    \\
     %               Метеоусловия		&       \\
     %               Покрытие			&       \\
                    \hline
                \end{tabular}\hfil}%
            \vspace{0.3cm}

            \hbox to 0.55\textwidth{\hfil%
                \begin{tabular}{|p{5cm}|>{\centering\arraybackslash}p{1.5cm}|}
                    \hline
                        &		\\			
                    \hline
                        &       \\
                        &		\\
                    \hline
                \end{tabular}\hfil}%
        }}%
    }
    \vspace{0.8cm}
    
%подъём 330

%5:13 --  5:54 (6:02) h=3250
%набираем по тропе к леднику Тютю(?) На тропе есть пара туров. На месте привала есть чистая вода.

%6:13 -- 6:23 дошли до ледника, связываемся

%6:52 -- 7:33 (7:38) h=3510 набираем в связках по леднику

%7:50 -- 8:29 (8:34) h=3710

%8:45 -- 9:21 (9:26) h=3870
%ещё ступень ледника; возможны ночёвки на скальном отрожке ~3850

%? -- 11:20 (11:37 1ая св) ледорубная дорожка h=3980 связка по связке -> ледобурн дорожка
%забрались на +1 ступень

%11:40 (11:55 1ая связка) -- 12:40 (12:48 2ая связка)
%почти дошли в связках на перевал. проходим берг со страховкой

%13:00 -- 13:15 (13:20) вылезли на не перевал не стоянку но обед

%+10 мин чхв до ночёвки

ЧХВ 1го: 41 + 10 + 41 + 39 + 36 + ??? + 45 + 15 + 10 = ??? + 3:57
ЧХВ группы: 49 + 10 + 46 + 44 + 41 + ??? + 68 + 20 + 10 = ??? + 4:48


    \FloatBarrier
