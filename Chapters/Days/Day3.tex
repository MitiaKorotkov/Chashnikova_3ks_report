\graphicspath{{Pictures/Chapter5/Day3}}

\subsection{День 3 (2 июля)}\label{subsec:Day3}
    \vbox{%
        \hbox to \textwidth{\hfil%
        \includegraphics[width=0.7\textwidth]{day3sm.pdf}\label{fig:Day3_map}\hfil}
        \vspace{0.3cm}
        
        \hbox to \textwidth{\hfil%
        \begin{tabular}{|p{4.5cm}|>{\centering\arraybackslash}p{4cm}|}
            \hline
            Расстояние, км      &   $11,5$          \\
            Набор высоты, м     &   $+950 / -270$   \\
            Высота ночёвки, м	&   $2517$          \\
            \hline
        \end{tabular}\quad%
        \begin{tabular}{|p{5cm}|>{\centering\arraybackslash}p{1.5cm}|}
            \hline
            ЧХВ за день &   8:10    \\			
            ГХВ за день &   9:35    \\
            \hline
        \end{tabular}\hfil}%
    }
    \vspace{0.8cm}

    Выход из турбазы Уштулу в 6:55. Идём по автомобильной дороге вниз до слияния рек Карасу и Дыхсу, затем дорога
    поворачивает направо. Примерно через 1,5\,км после поворота переходим Черек Балкарский по мосту. У моста начинается
    хорошая тропа, она забирает вверх и поворачивает в сторону долины Дыхсу. Затем тропа снова поворачивает и дальше мы
    идем по левому борту долины в сторону каньона реки Тютюнсу. Тропа постепенно становится менее заметной, заросшей.
    Идем в траве по пояс, иногда приходится продираться через кустарники и деревья. Жарко и влажно. В 9:20 выходим на
    огромную зеленую поляну рядом с впадением Тютюнсу в Черек Балкарский. Здесь хорошее место, чтобы поставить палатки
    и заночевать.

    Впереди виднеется лес, ребята уходят без рюкзаков в поисках начала тропы. Вроде какую-то тропу нашли, надеваем
    рюкзаки и идем за штурманом. Лес преимущественно хвойный, проходимый, много камней, мха, рыжих иголок. Явной тропы
    нет, иногда встречаем маркировку, идем по прямой линии на которой нет деревьев наверх. Встречаются скользкие камни,
    иногда путь идет рядом с крутыми обрывами: нужно быть аккуратными. В 10 утра хронометрист отмечает в блокнотике,
    что путь проходит через продолжительный бурелом. В 11 утра лес становится более редким, тропа иногда пропадает.
    В 12:40 выходим из леса. Начинаются огромные поля крупного курумника рядом с рекой, идём по правому орографически
    берегу. Туман, очень низкая видимость. Выходим вплотную к реке и идём вдоль неё. В 14:00 выходим к первому отмеченному
    месту ночевки (высота 2230\,м), обедаем. Место не очень комфортное для ночевки. В 15:20 выходим дальше, до слияния
    рукавов реки Тютюнсу, далее движемся вдоль левого пхд рукава. Путь пролегает по березовому мелколесью без тропы. Лес
    относительно неплохо проходим. Есть отличные места для ночевки на высоте около 2290\,м (фото 3.5).

    К 16:30 доходим до планового места ночевки, но заросли малопривлекательны. Более бодрая часть группы надеется сегодня
    выйти из леса и заночевать на высоте около 2550\,м. Продираемся через заросли ивы и рододендрона, идем по руслу реки.
    Русло достаточно узкое, берега крутые. После выхода к руслу темп группы катастрофически замедлился, все устали от
    пешеходного туризма. В 17:35 на высоте 2430\,м ровняем места под 2 палатки, встаём на ночевку.

    \FloatBarrier

    %1)  6:55 -- 7:42 (посл 7:45) Идём по дороге, вниз, далее переходим мост (пешеходный) через Черек-Баши, поъём по тропе , она забирает чуть вверх(?) назад по долине

    %2) 7:55 -- 8:34 (посл 8:37) 10 мин идём по исчезающей тропе / треку gps, далее выходим на тропу влесу. Заросли! Тропа немного отличается от карты.

    %3) 8:47 -- 9:21 (посл 9:22) привал недалеко от возможного места ночёвки, далее снова будет лес

    %4) 9:32 -- 10:09 (посл 10:12?) вынужденный привал -- Катя подскользнулась, обрабатываем ранку. Полная жопа бурелома.

    %5) 10:22 -- 10:59 (посл 11:03) h=1890 лес более редкий, тропа иногда пропадает

    %6) 11:15 -- 11:56 сели около камня

    %7) 12:10 -- 12:56 (посл 13:10) h=2150 в 12:40 вышли из леса на курумник, идём по правому берегу. туман, видимость 30м

    %8) 13:20 -- 14:13 (посл 14:16) выходим вплотную к реке и идём вдоль неё, привал на псевдостоянке. перекус-обед

    %9) 15:19 -- 16:04 (посл 16:08) идём до слияния реки с ручьём, далее по ручью, далее по берёзовой роще. h=2330

    %10) 16:19 -- 17:15 (посл 17:25) разведка Насти(?), идём дальше

    %17:35 -- h=2430 примерно

    %18:00 ровняем

    %чхв 1го за день 460 мин = 7:40
    %чхв группы за день 510 мин? = 8:30
    %гхв группы 9:35