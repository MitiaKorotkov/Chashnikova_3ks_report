\graphicspath{{Pictures/Chapter5/Day19}}

\fxnote{Сделать КАРТИНКУ В ТЕКСТЕ! то есть текст должен быть справа от карты в том числе}


\subsection{День 19 (18 июля)}\label{subsec:Day19}
    \parbox[c]{\textwidth}{%
        \includegraphics[width=0.4\textwidth]{day19.pdf}\label{fig:Day19_map}%
        \raisebox{\height}{\vbox{%
            \hbox to 0.55\textwidth{\hfil%
                \begin{tabular}{|p{4.5cm}|>{\centering\arraybackslash}p{2cm}|}
                    \hline
                    Расстояние, км		&  9.5  \\
                    Набор высоты, м		&  -800 \\
                    Высота ночёвки, м	&   1516 \\ %????
                    \hline
                \end{tabular}\hfil}%
            \vspace{0.3cm}

            \hbox to 0.55\textwidth{\hfil%
                \begin{tabular}{|p{3.5cm}|>{\centering\arraybackslash}p{1.5cm}|}
                   \hline
                      ЧХВ за день & 2:06 \\
                      ГХВ за день & 2:30      \\
                    \hline
                \end{tabular}\hfil}%
        }}%
    }
    \vspace{0.8cm}

Подъём в 4:30. Выходим из кемпинга в 6:45. За час дошли до родника с чистой водой, рядом есть возможные места ночёвки в лесу ближе к реке. Мимо проезжают погранцы, спрашивают откуда пришли, проверяют паспорта и пропуска (здесь 10-километровая погранзона). Спускаемся вдоль подъемника по лестнице и дальше по дороге выходим из каньона реки Адыр-Су, здесь нас должна забрать машина. В 9:15 садимся на рюкзаки и ждем водителя.

ПОХОД ОКОНЧЕН! ГРУППА СЧАСТЛИВА!

%подъём 430 (по факту позже)

%выход 6:45
%6:45 -- 7:33 (посл 7:34) дошли до родника с чистой водой, рядом есть возможные места ночёвки в лесу ближе к реке

%Табличка про нацпарк
%Идём по дороге в верхний баксан

%7:45 -- 8:30 (посл 8:32) дошли до очередной скамейки =)

%8:45 -- 9:15 посл ходка дод баксана

%ПОХОД ОКОНЧЕН!

%готово
%ЧХВ группы: 2:06
%ГХВ группы: 2:30
%ЧХВ первого: 2:03

    \FloatBarrier
