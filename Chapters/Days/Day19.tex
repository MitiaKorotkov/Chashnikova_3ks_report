\graphicspath{{Pictures/Chapter5/Day19}}

\subsection{День 19 (18 июля)}\label{subsec:Day19}
    \parbox[c]{0.95\textwidth}{%
        \includegraphics[width=0.4\textwidth]{day19.pdf}\label{fig:Day19_map}\hfill%
        \raisebox{0.45\height}{\vbox{%
            \hbox to 0.55\textwidth{\hfil%
                \begin{tabular}{|p{4.5cm}|>{\centering\arraybackslash}p{2cm}|}
                    \hline
                    Расстояние, км		&   $9,5$  \\
                    Набор высоты, м		&   $-800$ \\
                    Высота ночёвки, м	&   $1516$ \\% FIXME(): Оставить ли
                    \hline
                \end{tabular}\hfil}%
            \vspace{0.3cm}

            \hbox to 0.55\textwidth{\hfil%
                \begin{tabular}{|p{3.5cm}|>{\centering\arraybackslash}p{1.5cm}|}
                    \hline
                    ЧХВ за день & 2:06    \\
                    ГХВ за день & 2:30    \\
                    \hline
                \end{tabular}\hfil}%
            \vspace{1cm}

            \hbox to 0.55\textwidth{\hfil%
            \parbox[c]{0.46\textwidth}{%
            Подъём в 4:30. Выходим из кемпинга в 6:45. За час дошли до родника с чистой водой, рядом есть возможные места
            ночёвки в лесу ближе к реке. Мимо проезжают погранцы, спрашивают откуда пришли, проверяют паспорта и пропуска
            (здесь 10-километровая погранзона). Спускаемся вдоль подъемника по лестнице и дальше по дороге выходим из
            каньона реки Адыр-Су, здесь нас должна забрать машина. В 9:15 садимся на рюкзаки и ждем водителя.\hfil}\hfil}
        }}%
    }

    % Подъём в 4:30. Выходим из кемпинга в 6:45. За час дошли до родника с чистой водой, рядом есть возможные места
    % ночёвки в лесу ближе к реке. Мимо проезжают погранцы, спрашивают откуда пришли, проверяют паспорта и пропуска
    % (здесь 10-километровая погранзона). Спускаемся вдоль подъемника по лестнице и дальше по дороге выходим из
    % каньона реки Адыр-Су, здесь нас должна забрать машина. В 9:15 садимся на рюкзаки и ждем водителя.

    \FloatBarrier
    
    %подъём 430 (по факту позже)
    
    %выход 6:45
    %6:45 -- 7:33 (посл 7:34) дошли до родника с чистой водой, рядом есть возможные места ночёвки в лесу ближе к реке
    
    %Табличка про нацпарк
    %Идём по дороге в верхний баксан
    
    %7:45 -- 8:30 (посл 8:32) дошли до очередной скамейки =)
    
    %8:45 -- 9:15 посл ходка дод баксана
    
    %ПОХОД ОКОНЧЕН!
    
    %готово
    %ЧХВ группы: 2:06
    %ГХВ группы: 2:30
    %ЧХВ первого: 2:03