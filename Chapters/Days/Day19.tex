\graphicspath{{Pictures/Chapter5/Day19}}


\subsection{День 19 (18 июля)}\label{subsec:Day19}
    \parbox[c]{\textwidth}{%
        \includegraphics[width=0.4\textwidth]{day19.pdf}\label{fig:Day19_map}%
        \raisebox{\height}{\vbox{%
            \hbox to 0.55\textwidth{\hfil%
                \begin{tabular}{|p{4.5cm}|>{\centering\arraybackslash}p{4cm}|}
                    \hline
                    Расстояние, км		&    \\
                    Набор высоты, м		&   \\
                    Высота ночёвки, м	&    \\
                    Метеоусловия		&       \\
                    Покрытие			&       \\
                    \hline
                \end{tabular}\hfil}%
            \vspace{0.3cm}

            \hbox to 0.55\textwidth{\hfil%
                \begin{tabular}{|p{5cm}|>{\centering\arraybackslash}p{1.5cm}|}
                    \hline
                        &		\\			
                    \hline
                        &       \\
                        &		\\
                    \hline
                \end{tabular}\hfil}%
        }}%
    }
    \vspace{0.8cm}

подъём 430 (по факту позже)

выход 6:45
6:45 -- 7:33 (посл 7:34) дошли до родника с чистой водой, рядом есть возможные места ночёвки в лесу ближе к реке

Табличка про нацпарк
Идём по дороге в верхний баксан

7:45 -- 8:30 (посл 8:32) дошли до очередной скамейки =)

8:45 -- 9:15 посл ходка дод баксана

ПОХОД ОКОНЧЕН!

ЧХВ группы: 49 + 47 + 30 = 2:06

ЧХВ первого: 48 + 45 + 30 = 2:03

    \FloatBarrier