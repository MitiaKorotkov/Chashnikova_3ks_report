\graphicspath{Pictures/Chapter5/Day19}


\subsection{День 19 (18 июля)}\label{subsec:Day19}
    \begin{figure}[ht]
        \centering
        \includegraphics[width=0.6\textwidth, angle=90]{Pictures/Chapter5/Day19/day19.pdf}\label{fig:Day19_map}

        \begin{tabular}{|p{4.5cm}|>{\centering\arraybackslash}p{4cm}|}
            \hline
            Расстояние, км		&       \\
            Набор высоты, м		&       \\
            Высота ночёвки, м	&       \\
            Метеоусловия		&       \\
            Покрытие			&       \\
            \hline
        \end{tabular}\quad
        \begin{tabular}{|p{5cm}|>{\centering\arraybackslash}p{1.5cm}|}
            \hline
            	&		\\			
            \hline
                &       \\
                &		\\
            \hline
        \end{tabular}
    \end{figure}


подъём 430 (по факту позже)

выход 6:45
6:45 -- 7:33 (посл 7:34) дошли до родника с чистой водой, рядом есть возможные места ночёвки в лесу ближе к реке

Табличка про нацпарк
Идём по дороге в верхний баксан

7:45 -- 8:30 (посл 8:32) дошли до очередной скамейки =)

8:45 -- 9:15 посл ходка дод баксана

ПОХОД ОКОНЧЕН!

ЧХВ группы: 49 + 47 + 30 = 2:06

ЧХВ первого: 48 + 45 + 30 = 2:03

    \FloatBarrier