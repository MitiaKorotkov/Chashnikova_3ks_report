\graphicspath{{Pictures/Chapter5/Day15}}

\subsection{День 15 (14 июля)}\label{subsec:Day15}
    \vbox{%
        \hbox to \textwidth{\hfil%
        \includegraphics[width=0.6\textwidth]{day15.pdf}\label{fig:Day15_map}\hfil}
        \vspace{0.3cm}

        \hbox to \textwidth{\hfil%
        \begin{tabular}{|p{4.5cm}|>{\centering\arraybackslash}p{2.5cm}|}
            \hline
            Расстояние, км		&   $8,4$   \\
            Набор высоты, м		&   $+1480$ \\
            Высота ночёвки, м	&   $3574$  \\
            \hline
        \end{tabular}\quad%
        \begin{tabular}{|p{3.4cm}|>{\centering\arraybackslash}p{1.5cm}|}
            \hline
            ЧХВ за день &   5:51    \\
            ГХВ за день &   7:30    \\
            \hline
        \end{tabular}\hfil}%
    }
    \vspace{0.8cm}
    \fxnote{Сделать таблицы справа от карты компактненько?}

    Наша цель на сегодня "--- подойти под перевал Килар.

    В 06:10 выход, набираем высоту по достаточно крутой тропе в лесу, лес быстро заканчивается и выводит на
    пологую часть долины. Встречаем много коров и лошадей. В долине идет хорошая тропа. На слиянии рек Кенчат и
    Джайлык поворачиваем в долину реки Кенчат вместе с тропой, поворот маркирован турами. Главное не терять тропу,
    встречаем по пути семью в кроссовках и с маленьким ребенком.

    В 11:40 проходим красивые ночевки на высоте 3200\,м. Трава заканчивается, начинаются поля курумника. В 13:40
    доходим до верхних ночевок на высоте 3585\,м. Ровняем площадки, обсуждаем планы на завтра. Погода чудесная,
    весь день светит солнце.

    \FloatBarrier
    
    %подъём 4:00 (4:40 мы)

    %1) 06:10 -- 06:52 (06:54) h=2325 набираем по тропе, в лесу рядом с тропой есть трос на левый орогр берег реки джайлык(?)

    %2) 07:06 -- 07:44 (07:46) h=2500 продолжаем по тропе вверх, коровы и лошади

    %3) 07:58 -- 08:37 (08:44) повернули направо поворот маркирован турами h=2720

    %4) 08:58 -- 09:37 (09:50) h=2935 продолжаем набирать, тропа теряется, возможен путь по курумнику либо траверс трав склона

    %5) 10:01 -- 10:40 (10:55) h=3120 идём по дохлой тропе, она марки турами и идёт вдоль ручья. бахаем 2 порции гейнера

    %6) 11:25 -- 12:02 (12:11) h=3280 прошли красивые ночёвки на ~3200м. тропа всё ещё маркир турами, идём вдоль ручья

    %7) 12:23 -- 12:56 (13:00) путь маркир турами продолж по курумнику h=3450

    %8) 13:12 -- 13:40 (13:40) пришли на моренные стоянки на ~3585

    %готово
    %ЧХВ 1го: 4:59
    %ЧХВ группы: 5:51
    %ГХВ группы: 7:30