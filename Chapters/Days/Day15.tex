\graphicspath{{Pictures/Chapter5/Day15}}


\subsection{День 15 (14 июля)}\label{subsec:Day15}
    \vbox{%
        \hbox to \textwidth{\hfil%
        \includegraphics[width=0.7\textwidth]{day15.pdf}\label{fig:Day15_map}\hfil}
        \vspace{0.3cm}

        \hbox to \textwidth{\hfil%
        \begin{tabular}{|p{4.5cm}|>{\centering\arraybackslash}p{4cm}|}
            \hline
            Расстояние, км		&       \\
            Набор высоты, м		&       \\
            Высота ночёвки, м	&       \\
            Метеоусловия		&       \\
            Покрытие			&       \\
            \hline
        \end{tabular}\quad%
        \begin{tabular}{|p{5cm}|>{\centering\arraybackslash}p{1.5cm}|}
            \hline
                &		\\			
            \hline
                &       \\
                &		\\
            \hline
        \end{tabular}\hfil}%
    }
    \vspace{0.8cm}
    
подъём 4:00 (4:40 мы)

1) 06:10 -- 06:52 (06:54) h=2325 набираем по тропе, в лесу рядом с тропой есть трос на левый орогр берег реки джайлык(?)

2) 07:06 -- 07:44 (07:46) h=2500 продолжаем по тропе вверх, коровы и лошади

3) 07:58 -- 08:37 (08:44) повернули направо поворот маркирован турами h=2720

4) 08:58 -- 09:37 (09:50) h=2935 продолжаем набирать, тропа теряется, возможен путь по курумнику либо траверс трав склона

5) 10:01 -- 10:40 (10:55) h=3120 идём по дохлой тропе, она марки турами и идёт вдоль ручья. бахаем 2 порции гейнера

6) 11:25 -- 12:02 (12:11) h=3280 прошли красивые ночёвки на ~3200м. тропа всё ещё маркир турами, идём вдоль ручья

7) 12:23 -- 12:56 (13:00) путь маркир турами продолж по курумнику h=3450

8) 13:12 -- 13:40 (13:40) пришли на моренные стоянки на ~3585


ЧХВ группы: 44 + 44 + 46 + 52 + 54 + 46 + 37 + 28 = 5:51

ЧХВ первого: 42 + 42 + 39 + 39 + 39 + 37 + 33 + 28 = 4:59

    \FloatBarrier
