\graphicspath{{Pictures/Chapter5/Day4}}


\subsection{День 4 (3 июля)}\label{subsec:Day4}
    \vbox{%
        \hbox to \textwidth{\hfil%
        \includegraphics[width=0.7\textwidth]{day4.pdf}\label{fig:Day4_map}\hfil}
        \vspace{0.3cm}
        
        \hbox to \textwidth{\hfil%
        \begin{tabular}{|p{4.5cm}|>{\centering\arraybackslash}p{4cm}|}
            \hline
            Расстояние, км		&    4.5   \\
            Набор высоты, м		&   +1181/-227    \\
            Высота ночёвки, м	&    3417   \\
            Метеоусловия		&       \\
            Покрытие			&       \\
            \hline
        \end{tabular}\quad%
        \begin{tabular}{|p{5cm}|>{\centering\arraybackslash}p{1.5cm}|}
            \hline
                &		\\			
            \hline
                &       \\
                &		\\
            \hline
        \end{tabular}\hfil}%
    }
    \vspace{0.8cm}

Подъём 5
1)  7:04 -- 7:45 (посл 7:54) Сушили вещи, тупим после вчера h=2720. Поднимались по руслу, от слияния ручьёв -- по орогр правому
Выходим в висячую долину Тютюсу(?) Немного вчера не дошли до ровной травы

2) 8:05 -- 8:46 (посл 8:57) h=2950. Идём по руслу ручья, по нему выходим на следующую ступень доллины. Ручей уходит под снежник посередине ходки, идём по этому зачехлённому снежнику. На небе переменная облачность чутка

3) 9:09 -- 9:44 (посл 9:58) h=3140, идём по грязному снежнику не доходя до конца, уходим направо пхд по ручью , потом по его левому орографически берегу по заросшему гребню. На гребне тропа (к след ночёвкам)

4) 10:10 -- 10:54 (посл 11:08) h=3350 Идём по снежнику, далее выходим на левую орогр моренку, проходим её и далее тропим. Глубина ступеней небольшая (можно без гамаш) Кинулись на моренке, впереди полка

5) 11:18 -- 11:54 (посл 12:00) h=3460 Тропим дальше, максимально по моренкам идём, почти у полки

6) 12:12 -- 13:05 (посл 13:20) тропим до перевала, ставим тент, погода -- жопа, туман, снегодождь, облако. На перевале обед, уходим с него в 15:45


Движемся в связках, спуск до морены к 16:45.
Настя ищет площадки, ровняем их. В 17:00 все встали, отбой в ~20. Подхём завтра в 4:30.



    \FloatBarrier
