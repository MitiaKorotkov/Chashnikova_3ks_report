\graphicspath{{Pictures/Chapter5/Day4}}

\subsection{День 4 (3 июля)}\label{subsec:Day4}
    \vbox{%
        \hbox to \textwidth{\hfil%
        \includegraphics[width=0.55\textwidth, angle=90]{day4sm.pdf}\label{fig:Day4_map}\hfil}
        \vspace{0.3cm}

        \hbox to \textwidth{\hfil%
        \begin{tabular}{|p{4.5cm}|>{\centering\arraybackslash}p{4cm}|}
            \hline
            Расстояние, км		&   $4,4$           \\
            Набор высоты, м		&   $+1140 / -230$  \\
            Высота ночёвки, м	&   $3428$          \\
            \hline
        \end{tabular}\quad%
        \begin{tabular}{|p{5cm}|>{\centering\arraybackslash}p{1.5cm}|}
            \hline
            ЧХВ за день &   6:24    \\		
            ГХВ за день &   7:21    \\		
            \hline
            ЧХВ от МН до \newline пер.\,Туристов Грузии & 5:19    \\ 
            \hline
            ЧХВ от пер.\,Туристов Грузии до МН          & 1:05    \\
            \hline
        \end{tabular}\hfil}%
        }
    \vspace{0.8cm}

    Подъём в 5:00, выход в 7:05.

    Поднимаемся по руслу ручья в висячую долину реки Тютюнсу. В верховьях ручья лежит снежник, идём по нему, затем
    забираем правее и выходим на гребни заросших моренных холмов. Гребень заканчивается, снова выходим на снежник,
    со снежника правее на осыпной гребень, все время забираем вправо, вдоль выходов скал. Подходим к северо-восточному
    перевальному взлету. Путь подъема по пологому снежнику на наклонной полке. Крутизна снежного склона до $25^\circ$
    и не требует движения в кошках. Поднимаемся в ботинках, делаем ступени, индивидуальная самостраховка ледорубом.
    Со скал при нас ничего не сыпалось, но есть следы от камней длиной в несколько метров, пересекающие путь подъёма.
    Этот участок проходим быстро, по одному. В 13:20 собираемся на перевале. Обедаем, пережидаем непогоду. Седловина
    перевала широкая, снежно-осыпная, здесь вполне можно разбить лагерь. В 15:45 начинаем спуск с ледника, видимость
    около 30\,м. Спуск осуществляем в связках, крутизна склона примерно до 15--20 градусов, преимущественно ледник
    закрытый.

    Спускаемся на морены в 16:45. Идем на разведку площадок, в 17:00 группа собирается вместе на месте ночевки.

    \FloatBarrier
    
    %далее оригинал хронометража

    %Подъём 5
    %1)  7:04 -- 7:45 (посл 7:54) Сушили вещи, тупим после вчера h=2720. Поднимались по руслу, от слияния ручьёв -- по орогр правому
    %Выходим в висячую долину Тютюсу(?) Немного вчера не дошли до ровной травы

    %2) 8:05 -- 8:46 (посл 8:57) h=2950. Идём по руслу ручья, по нему выходим на следующую ступень доллины. Ручей уходит под снежник посередине ходки, идём по этому зачехлённому снежнику. На небе переменная облачность чутка

    %3) 9:09 -- 9:44 (посл 9:58) h=3140, идём по грязному снежнику не доходя до конца, уходим направо пхд по ручью , потом по его левому орографически берегу по заросшему гребню. На гребне тропа (к след ночёвкам)

    %4) 10:10 -- 10:54 (посл 11:08) h=3350 Идём по снежнику, далее выходим на левую орогр моренку, проходим её и далее тропим. Глубина ступеней небольшая (можно без гамаш) Кинулись на моренке, впереди полка

    %5) 11:18 -- 11:54 (посл 12:00) h=3460 Тропим дальше, максимально по моренкам идём, почти у полки

    %6) 12:12 -- 13:05 (посл 13:20) тропим до перевала, ставим тент, погода -- жопа, туман, снегодождь, облако. На перевале обед, уходим с него в 15:45


    %Движемся в связках, спуск до морены к 16:45.
    %Настя ищет площадки, ровняем их. В 17:00 все встали, отбой в ~20. %Подъём завтра в 4:30.
    %готово
    %чхв 1го до перевала 250 мин = 4:10
    %чхв группы до перевала 319 мин = 5:19
    %гхв группы до перевала 376 мин = 6:16

    %чхв 1го после перевала = 60 мин = 1:00
    %чхв группы после перевала = 65 мин = 1:05 (примерно)
    %гхв после перевала = 1:05

    %чхв 1го за день = 5:10
    %чхв группы за день = 6:24
    %гхв группы за день = 7:21
