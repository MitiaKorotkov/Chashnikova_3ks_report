\graphicspath{{Pictures/Chapter5/Day2}}

\subsection{День 2 (1 июля)}\label{subsec:Day2}
    \begin{figure}[h]
    \fxwarning{картинка плохо стоит}
        \centering
        \includegraphics[width=0.6\textwidth, angle=90]{Pictures/Chapter5/Day2/day2.pdf}\label{fig:Day2_map}

        \begin{tabular}{|p{4.5cm}|>{\centering\arraybackslash}p{4cm}|}
            \hline
            Расстояние, км		&    20.3   \\
            Набор высоты, м		&    +1140/-1940   \\
            Высота ночёвки, м	&    1843   \\
            Метеоусловия		&       \\
            Покрытие			&       \\
            \hline
        \end{tabular}\quad
        \begin{tabular}{|p{5cm}|>{\centering\arraybackslash}p{1.5cm}|}
            \hline
            	&		\\			
            \hline
                Итого ЧХВ							&		\\
            Итого ГХВ							&		\\
            \hline
        \end{tabular}
    \end{figure}

Выходим в 5:50. Переходим речку и на правом берегу ставим палатки сушиться (высота 2550), оставляем в них рюкзаки. Продолжаем путь налегке. Около 7 утра пошёл дождь, надеваем мембраны. Переходим на орографически левый берег ручья около снежного моста, идём плотной группой, т.к. начался туман.
При подъёме на перевал 1-ый по ходу движения снежник обходим слева; 2-ой по ходу движения снежник (перед взлётом) стоило проходить по снегу, мы частично обошли. Вместо привалов делаем К 9:10 вся группа собралась на перевале, фотографируем записку и оставляем её следующей прямо за нами группе из горной секции МФТИ.

далее оригинал хронометража

1) 5:50 -- 6:15 (посл 6:25) Переходим речку и ставим палатки сушиться, кидаем в них рюкзаки. На правом берегу. h = 2550м

2) 6:41 -- 7:15 (посл 7:18) Пошёл дождь, надеваем мембраны h = 2830м.

3) 7:28 -- 8:09 (посл 8:15) Переходим около снежного моста на орографически левый берег ручья, идём на пер.  Туман, идём плотно. h = 3100.

4) 8:16 -- 9:05 (посл 9:10) перевал; фоткаем записку, оставляем её следующей группе <<горносеков>> h = 3330.

Подъём: 1 пхд снежник обходим слева, 2ой снежник перед взлётом стоило проходить по снегу, мы частично обошли. далее туман, вместо привалов делаем сборы

5) 9:18 -- 10:15 (посл?) вершина(?), тур без записки. 2ой жандарм обходим слева пхд по снежнику. Мокрый снег, туман. Даша и Дима остались в низу подъёма.

6) 10:32 -- 10:47 (h=3410) спустились до Даши и Димы, они построили каменную стенку от ветра (h=3327) В 10:58 1-ый спустил до перевала. пишем записку.

7) 11:10 -- 11:20. Короткая ходка, привал на разутепление.

В сторону вершины поъём был по мелкой подвижной осыпи <18 градусов уклона. Выступ рыжих скал обходим справа пхд по тропе, а 2ой -- см раньше.

На пути 3-4 снежника. Сне мокрый, ступени делаются легко. Нужно быть аккуратным, со снежников плохой выкат. Движение по снежникам <= 13 градусов. Самостраховка ледорубом. Перевальный взлёт с западной стороны осыпной, осыпь мелкая, подвижная перед перевальным взлётом, пологий снежник <10 градусов утром подмёрз. Движение по нему без затруднений.

8) 11:30 -- 12:25 (посл 12:32) Спустили до палаток, готовим обед.

9) 14:10 -- 14:55 (14:57 посл) перешли речку, теперь по "мосту". Я в ботинках пешком

10) 15:07 -- 16:10 (посл 16:11) Подъём по тропе, потом по дороге

к 17:40 пришли на ночёвку


    \FloatBarrier
