\graphicspath{{Pictures/Chapter5/Day2}}

\subsection{День 2 (1 июля)}\label{subsec:Day2}
    \begin{figure}[h]
    \fxwarning{картинка плохо стоит}
        \centering
        \includegraphics[width=0.6\textwidth, angle=90]{Pictures/Chapter5/Day2/day2.pdf}\label{fig:Day2_map}

        \begin{tabular}{|p{4.5cm}|>{\centering\arraybackslash}p{4cm}|}
            \hline
            Расстояние, км		&    20.3   \\
            Набор высоты, м		&    +1140/-1940   \\
            Высота ночёвки, м	&    1843   \\
            \hline
        \end{tabular}\quad
        \begin{tabular}{|p{5cm}|>{\centering\arraybackslash}p{1.5cm}|}
            \hline
            	&		\\			
            \hline
            ЧХВ 1-го за день							&		\\
            ГХВ группы за день							&		\\ \hline
        \end{tabular}
    \end{figure}

Выход в 5:50. Возвращаемся в долину реки Карасу. Переодеваемся в бродную обувь, переходим разливы ручьев и ставим палатки сушиться на правом берегу (высота 2550). Палатки сырые из-за конденсата, оставляем их сушиться, внутрь палатки складываем вещи. Погода туманная с самого утра, кажется, скоро начнется дождь. В 6:40 выходим в направлении перевала Штулу. В 7:00 начинается дождь. Подходим к холму, по гребню которого идет хорошая тропа. Эта тропа идет почти до самого перевала, но иногда становится слабо заметной и теряется. Идем по зеленым полям, встречаются поля цветущих рододендронов. Приток реки Карасу переходим около снежного моста на орографически левый берег на высоте около 3000м, движемся в сторону перевала. Рельеф простой: травянистые холмы, размытые дождем, поля мелкой осыпи, небольшие участки пологих снежников.  Перевальный взлет представляет собой мелкоосыпной склон в котором легко оставлять ступени, до 20 градусов по крутизне. В 9:10 группа собирается на перевале. Принимаем решение дойти до вершины Штулутау для лучшей акклиматизации, несмотря на туман: рельеф простой, ориентируемся по gps-треку.

В 9:20 выдвигаемся к вершине Штулутау. Один участник идет очень медленно, болит голова, по симптомам похоже на горняшку. Оставляем 2 участников на одном из выполаживаний, отдаем им лишнюю теплую одежду и рацию.

Путь к вершине представлен сланцевой осыпью небольшой крутизны (до 20 градусов) и не вызывает сложностей.  Выступ рыжих скал обходим справа пхд по тропе, второй черный жандарм слева пхд по снежнику. На снежный гребень перед вершиной не выходим, а обходим снежники слева пхд, иногда приходится пересекать короткие пологие снежные участки. Вершина осыпная, собираемся на ней в 10:15, находим тур без записки, оставляем свою, фотографируемся и начинаем спуск. На спуске встречаемся с горнящащимися участниками, они построили каменную стенку от ветра. В 11:00 собираемся на перевале, пишем записку и начинаем спуск по пути подъема.
В 12:30 собираемся у палаток, готовим обед. Выдвигаемся к турбазе Уштулу в 14:10. В этот раз зашли к нарзану, попили из него воды. Вода вкусная, но пахнет неприятно. В 17:40 пришли на ночёвку в турбазу.

%далее оригинал хронометража

%1) 5:50 -- 6:15 (посл 6:25) Переходим речку и ставим палатки сушиться, кидаем в них рюкзаки. На правом берегу. h = 2550м

%2) 6:41 -- 7:15 (посл 7:18) Пошёл дождь, надеваем мембраны h = 2830м.

%3) 7:28 -- 8:09 (посл 8:15) Переходим около снежного моста на орографически левый берег ручья, идём на пер.  Туман, идём плотно. h = 3100.

%4) 8:16 -- 9:05 (посл 9:10) перевал; фоткаем записку, оставляем её следующей группе <<горносеков>> h = 3330.

%Подъём: 1 пхд снежник обходим слева, 2ой снежник перед взлётом стоило проходить по снегу, мы частично обошли. далее туман, вместо привалов делаем сборы

%5) 9:18 -- 10:15 (посл?) вершина(?), тур без записки. 2ой жандарм обходим слева пхд по снежнику. Мокрый снег, туман. Даша и Дима остались в низу подъёма.

%6) 10:32 -- 10:47 (h=3410) спустились до Даши и Димы, они построили каменную стенку от ветра (h=3327) В 10:58 1-ый спустил до перевала. пишем записку.

%7) 11:10 -- 11:20. Короткая ходка, привал на разутепление.

%В сторону вершины поъём был по мелкой подвижной осыпи <18 градусов уклона. Выступ рыжих скал обходим справа пхд по тропе, а 2ой -- см раньше.

%На пути 3-4 снежника. Сне мокрый, ступени делаются легко. Нужно быть аккуратным, со снежников плохой выкат. Движение по снежникам <= 13 градусов. Самостраховка ледорубом. Перевальный взлёт с западной стороны осыпной, осыпь мелкая, подвижная перед перевальным взлётом, пологий снежник <10 градусов утром подмёрз. Движение по нему без затруднений.

%8) 11:30 -- 12:25 (посл 12:32) Спустили до палаток, готовим обед.

%9) 14:10 -- 14:55 (14:57 посл) перешли речку, теперь по "мосту". Я в ботинках пешком

%10) 15:07 -- 16:10 (посл 16:11) Подъём по тропе, потом по дороге

%к 17:40 пришли на ночёвку


ЧХВ 1го до перевала 2:29
чхв группы до перевала 2:50

    \FloatBarrier
