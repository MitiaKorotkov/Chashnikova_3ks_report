\graphicspath{{Pictures/Chapter5/Day2}}

\subsection{День 2 (1 июля)}\label{subsec:Day2}
    \vbox{%
        \hbox to \textwidth{\hfil%
        \includegraphics[width=0.6\textwidth, angle=90]{day2.pdf}\label{fig:Day2_map}\hfil}
        \vspace{0.3cm}

        \hbox to \textwidth{\hfil%
        \begin{tabular}{|p{4.5cm}|>{\centering\arraybackslash}p{4cm}|}
            \hline
            Расстояние, км		&    20.1 (в зачёт 3.9)   \\
            Набор высоты, м		&    +1000/-1950   \\
            Высота ночёвки, м	&    1848   \\
            \hline
        \end{tabular}\quad%
        \begin{tabular}{|p{5cm}|>{\centering\arraybackslash}p{1.5cm}|}
            \hline
            ЧХВ за день	&	8:26	\\			
            ГХВ за день       &         10:12      \\
            \hline
            ЧХВ рад. выхода на п. Штулу-Тау (до м. обеда)	&	5:30	\\
            ЧХВ от м. обеда до МН		          &	2:56     \\
            		&		\\ \hline
        \end{tabular}\hfil}%
    }
    \vspace{0.8cm}

Выход в 5:50. Возвращаемся в долину реки Карасу. Переодеваемся в бродную обувь, переходим разливы ручьев и ставим палатки сушиться на правом берегу (высота 2550). Палатки сырые из-за конденсата, оставляем их сушиться, внутрь палатки складываем вещи. Погода туманная с самого утра, кажется, скоро начнется дождь. В 6:40 выходим в направлении перевала Штулу. В 7:00 начинается дождь. Подходим к холму, по гребню которого идет хорошая тропа. Эта тропа идет почти до самого перевала, но иногда становится слабо заметной и теряется. Идем по зеленым полям, встречаются поля цветущих рододендронов. Приток реки Карасу переходим около снежного моста на орографически левый берег на высоте около 3000м, движемся в сторону перевала. Рельеф простой: травянистые холмы, размытые дождем, поля мелкой осыпи, небольшие участки пологих снежников.  Перевальный взлет представляет собой мелкоосыпной склон в котором легко оставлять ступени, до 20 градусов по крутизне. В 9:10 группа собирается на перевале. Принимаем решение дойти до вершины Штулутау для лучшей акклиматизации, несмотря на туман: рельеф простой, ориентируемся по gps-треку.

В 9:20 выдвигаемся к вершине Штулутау. Один участник идет очень медленно, болит голова, по симптомам похоже на горняшку. Оставляем 2 участников на одном из выполаживаний, отдаем им лишнюю теплую одежду и рацию.

Путь к вершине представлен сланцевой осыпью небольшой крутизны (до 20 градусов) и не вызывает сложностей.  Выступ рыжих скал обходим справа пхд по тропе, второй черный жандарм слева пхд по снежнику. На снежный гребень перед вершиной не выходим, а обходим снежники слева пхд, иногда приходится пересекать короткие пологие снежные участки. Вершина осыпная, собираемся на ней в 10:15, находим тур без записки, оставляем свою, фотографируемся и начинаем спуск. На спуске встречаемся с горнящащимися участниками, они построили каменную стенку от ветра. В 11:00 собираемся на перевале, пишем записку и начинаем спуск по пути подъема.
В 12:30 собираемся у палаток, готовим обед. Выдвигаемся к турбазе Уштулу в 14:10. В этот раз зашли к нарзану, попили из него воды. Вода вкусная, но пахнет неприятно. В 17:40 пришли на ночёвку в турбазу.

%готово
%ЧХВ 1го до перевала 2:29
%чхв группы до перевала 2:50

%чхв 1го от перевала до вершины, а потом палаток 147 мин
%чхв группы от перевала до вершины, а потом палаток 160 мин

%чхв 1го от обеда до ночёвки ~168 мин = 2:48
%чхв группы от обеда до ночёвки ~176 мин = 2:56

%чхв 1го за день 464 мин = 7:44
%чхв группы за день 506 мин = 8:26

%гхв группы за день: 612 мин = 10:12



    \FloatBarrier
