\graphicspath{{Pictures/Chapter5/Day9}}


\subsection{День 9 (8 июля)}\label{subsec:Day9}
    \vbox{%
        \hbox to \textwidth{\hfil%
        \includegraphics[width=0.7\textwidth, angle=90]{day9.pdf}\label{fig:Day9_map}\hfil}
        \vspace{0.3cm}

        \hbox to \textwidth{\hfil%
        \begin{tabular}{|p{4.5cm}|>{\centering\arraybackslash}p{4cm}|}
            \hline
            Расстояние, км		&   7.8    \\
            Набор высоты, м		&   -650    \\
            Высота ночёвки, м	&      2146 \\
    %        Метеоусловия		&       \\
    %        Покрытие			&       \\
            \hline
        \end{tabular}\quad%
        \begin{tabular}{|p{5cm}|>{\centering\arraybackslash}p{1.5cm}|}
           \hline
              ЧХВ за день  &  2:22     \\
              ГХВ за день  &	2:42  \\
            \hline
        \end{tabular}\hfil}%
    }
    \vspace{0.8cm}

Сегодня поздний подъем, встаем в 8:00. План -- дойти до альплагеря Безенги и устроить хорошую дневку. Завтракаем, сушим вещи, погода снова не радует. По отличной тропе, маркированной турами через ледник выходим на тропу, ведущую к альплагерю. Вышли в 9:30, в 12:30 пришли в альплагерь. Здесь мы сделали кучу дел: забрали заброску, помылись, початились, поели хычинов и выпили пива. Все счастливы и готовы к новым подвигам.

%свободный подъём, но не позже 800
%завтракаем, сушим вещи, облачность увеличивается
%1) 933 -- 1032 (посл 1036) h=2470 от баранкоша спуск к леднику, маркирован турами. идём по зачехлённому леднику вниз. погода -- облачно и микродождь

%2) 1046 -- 1137 (посл 1138) h=2170 продолжаем идти по тропе к а/л

%3) 1148 -- 1213 (посл 1215) пришли в ал, полудневка

%готово
%чхв 1го = 135 мин = 2:15
%чхв группы = 142 мин = 2:22
%гхв группы за день = 162 мин = 2:42

    \FloatBarrier
