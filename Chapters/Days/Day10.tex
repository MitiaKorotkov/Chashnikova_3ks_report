\graphicspath{{Pictures/Chapter5/Day10}}


\subsection{День 10 (9 июля)}\label{subsec:Day10}
    \vbox{%
        \hbox to \textwidth{\hfil%
        \includegraphics[width=0.7\textwidth]{day10.pdf}\label{fig:Day10_map}\hfil}
        \vspace{0.3cm}
        
        \hbox to \textwidth{\hfil%
        \begin{tabular}{|p{4.5cm}|>{\centering\arraybackslash}p{4cm}|}
            \hline
            Расстояние, км		&       \\
            Набор высоты, м		&       \\
            Высота ночёвки, м	&       \\
 %           Метеоусловия		&       \\
 %           Покрытие			&       \\
            \hline
        \end{tabular}\quad%
        \begin{tabular}{|p{5cm}|>{\centering\arraybackslash}p{1.5cm}|}
            \hline
                &		\\			
            \hline
                &       \\
                &		\\
            \hline
        \end{tabular}\hfil}%
    }
    \vspace{0.8cm}

Выходим из альплагеря в 5:45. Спускаемся от альплагеря к мосту, переходим Черек Безенгийский и идём по тропе по правому берегу реки вверх, очень скоро тропа теряется. Поднимаемся вверх по траве. Раньше здесь был забег <<вертикальный километр>>, сейчас от него остались только ржавые таблички, отмечающие каждые 100 м по высоте.

К 10 -- 11 утра идти становится проще, выходим со склона на травянистый гребень. Собираться на привалах на склоне неудобно, нужно быть аккуратными и не упустить рюкзак вниз.

К 12:30 поднимаемся до стоянок с пересохшим ручьем. Много мест под палатки, ручей слышно, но не видно. Уходим на разведку, но вблизи ручья не обнаруживаем. Решаем отправить участника с рацией дальше по пути подъема и обнаруживаем хорошую полку с ручьем в 5 минутах ходьбы от первых стоянок. В 13:00 собираемся на месте ночевки, ставим лагерь.

%1) 545 -- 630 (посл 633)

%2) 643 -- 712 (посл 729) h=2400 спускаемся от ал к мосту, переходим и идём по тропе по правому берегу реки вверх, далее по траве вверх

%3) 740 -- 810 (посл 835) h=2590

%4) 845 -- 915 (посл 945) h=2740

%5) 1005 -- 1037 (посл 1044) у Стефы улетел рюкзак перед последней ходкой (этой) вниз на 200м, Лёша идет доставать, челночим рюкзаки. Обшаемся с ним по рации. в конце привала видим подним Лёшу

%6) 1057 -- 1129 (посл 1137) привал на вершине гребня / кулуара тут дожидаемся Лёшу h=3025

%7) 1147 -- 1229 (gjck 1238) поднимаемся до стоянок с пересохшим ручьём. конец пути, по осыпи, маркирован турами h=3210

%8) ещё 5 мин чхв до стоянки с ручьём (см отчёт Шабалина)

%готово
за день
чхв 1го 245 = 4:05
чхв группы 339 = 5:39
гхв группы 404 = 6:44
\FloatBarrier
