\graphicspath{{Pictures/Chapter5/Day10}}

\subsection{День 10 (9 июля)}\label{subsec:Day10}
    \vbox{%
        \hbox to \textwidth{\hfil%
        \includegraphics[width=0.65\textwidth]{day10sm.pdf}\label{fig:Day10_map}\hfil}
        \vspace{0.3cm}
        
        \hbox to \textwidth{\hfil%
        \begin{tabular}{|p{4.5cm}|>{\centering\arraybackslash}p{4cm}|}
            \hline
            Расстояние, км		&   $3,5$           \\
            Набор высоты, м		&   $+1200 / -100$  \\
            Высота ночёвки, м	&   $3236$          \\
            \hline
        \end{tabular}\quad%
        \begin{tabular}{|p{3cm}|>{\centering\arraybackslash}p{1.5cm}|}
           \hline
            ЧХВ за день   &    5:39    \\
            ГХВ за день   &    6:44    \\
            \hline
        \end{tabular}\hfil}%
    }
    \vspace{0.8cm}

    Выходим из альплагеря в 5:45. Спускаемся от альплагеря к мосту, переходим Черек Безенгийский и идём по тропе
    по правому берегу реки вверх, очень скоро тропа теряется. Поднимаемся вверх по траве. Раньше здесь был забег
    <<вертикальный километр>>, сейчас от него остались только ржавые таблички, отмечающие каждые 100\,м по высоте.

    К 10--11 утра идти становится проще, выходим со склона на травянистый гребень. Собираться на привалах на
    склоне неудобно, нужно быть аккуратными и не упустить рюкзаки вниз.

    К 12:30 поднимаемся до стоянок с пересохшим ручьем. Много мест под палатки, ручей слышно, но не видно. Уходим
    на разведку, но вблизи ручья не обнаруживаем. Идем на разведку полки, о которой писал Шабалин в отчете,
    обнаруживаем ее в 5 минутах ходьбы от первых стоянок. Здесь совсем рядом со стоянкой в камнях течет ручей.
    В 13:00 собираемся на месте ночевки, ставим лагерь.

    \FloatBarrier
    
    %за день
    %чхв 1го 245 = 4:05
    %чхв группы 339 = 5:39
    %гхв группы 404 = 6:44