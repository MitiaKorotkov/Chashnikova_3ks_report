\graphicspath{{Pictures/Chapter5/Day12}}


\subsection{День 12 (11 июля)}\label{subsec:Day12}
    \parbox[c]{\textwidth}{%
        \includegraphics[width=0.43\textwidth]{day12sm.pdf}\label{fig:Day12_map}%
        \raisebox{\height}{\vbox{%
            \hbox to 0.55\textwidth{\hfil%
                \begin{tabular}{|p{4.5cm}|>{\centering\arraybackslash}p{2.6cm}|}
                    \hline
                    Расстояние, км	& 5.4   \\
                    Набор высоты, м	& +1060/-340  \\
                    Высота ночёвки, м	&  4018  \\
    %                Метеоусловия		&       \\
    %                Покрытие			&       \\
                    \hline
                \end{tabular}\hfil}%
            \vspace{0.3cm}

            \hbox to 0.55\textwidth{\hfil%
                \begin{tabular}{|p{7cm}|>{\centering\arraybackslash}p{1.4cm}|}
                    \hline
                        ЧХВ за день &	7:26	\\		
                        ГХВ за день &	9:51	\\
                    \hline
                        ЧХВ от МН до пер. Тютютргу &    2:26   \\
                        ЧХВ от пер. Тютюргу до МН &	5:00	\\
                    \hline
                \end{tabular}\hfil}%
        }}%
    }
    \vspace{0.8cm}
\fxnote{Сделать КАРТИНКУ В ТЕКСТЕ! то есть текст должен быть справа от карты в том числе}

Вышли в 5:20. Движемся вдоль реки Тютюргу. Сначала по тропе, потом тропа теряется. Заворачиваем правее пхд на язык ледника Булунгу. Ледник сильно зачехленный, поднимаемся в ботинках по камушкам. После выходим на выполаживание, ледник открытый, идем в кошках. Перед гребнем небольшой снежный взлет крутизной до 25 градусов, сам гребень представляет собой широкие осыпные поля. По дороге к туру есть хорошие места для ночевки с водой. В 8:20 собираемся на перевале Тютюргу.

В 8:40 начинаем спуск с перевала по крупной осыпи, движемся довольно медленно. Крупная осыпь сменяется подвижной мелкой осыпью, спускаемся плотной группой.

Собираемся группой внизу под перевалом у ледника Тютюргу в 10:45, делаем обед.

После обеда в 12:05 выходим в связках к перевалу Шаурту. Ледник закрытый, но слой снега неглубокий. Видим большую зону трещин, обходим ее слева пхд. Далее ледник плоский, перед перевалом еще один небольший взлет с трещинами, его обходим слева пхд. Крутизна склона до $30^\circ$.

В 14:00 подходим под перевальный взлет. Подъем на перевал Шаурту возможен несколькими способами: напрямую на седловину через бергшрунд или на гребень по разрушенным скалам, траектории подъема на гребень могут отличаться. При подъеме через вторую зону трещин встряли в тропежку по колено - решаем не идти напрямую на перевал по снежному склону, а подняться на скальный гребень. Выбираем достаточно простой путь подъема по осыпным кулуарам. При подъеме провесили 3 веревки - 2 траверс через осыпь и выходы скальных плит, 1 по осыпному кулуару на гребень. Станции организовали на скальных выступах. Осыпь очень подвижная, встречаются участки простого лазанья на протяжении всего подъема. Крутизна склона до $35^\circ$. Последний кулуар, по которому поднимались, сложен очень подвижной осыпью, на которой лежат крупные блоки. Не рекомендуем подниматься по нему плотной группой с ледника, так как подъем достаточно протяженный, а осыпь сильно едет.

В 16:40 собираемся на гребне, проходим еще 50-100 м до ночевок. Успеваем поставить лагерь и убрать вещи до сильного дождя, сильный ливень с грозой очень быстро загоняет нас в палатки.

%подъём 330, выход 521

%1) 521 -- 551 (посл 555) набираем воды из снежника, идём вдоль реки

%2) 605 -- 655 (посл 659) прошли зачехл часть ледника (налево от ручья), дошли до открытой части ледника Надеваем обвязки, кошки h=3630

%3) 729 -- 810 (посл 827) идём в кошках ледник, уходим налево слегка, последние 15м до осыпи тыкаем палкой закрытый ледник и выходим на осыпь
%по дороге к туру есть стоянки с водой (почти на перевале)

%4) 840 - ? (посл 950) спуск с перевала по осыпи, крупной

%Даша ударяет руку, оказываем помощь

%5) 1000 -- 1045 (посл 1050) спускаемся пао лифтам и живой осыпи до морены ледника (?) обед h=3500

%6) 1205 -- 1230 (?) прошлись в кошках до закрытой части ледника

%7) 1240 -- 1315 (посл 1320) идём в связках на следующую ступень ледника. Пересекаем 2 области закрытого ледника. заходим наверх слева пхд

%1335 -- 1407 = 32

%1412 -- 1450 = 38

%1500 -- 1530 = 30

%1540 -- 1610 = 30

%1620 -- 1640 = 20

%готово
%чхв 1го до перевала 121 = 2:01
%чхв группы до перевала 146 = 2:26
%гхв группы до перевала 186 = 3:06


%чхв 1го от перевала до ночёвки 100 + (трек) 150 = 250 = 4:10
%чхв группы от перевала до ночёвки + 15 + 5 + 0 + 5 = 300 = 5:00
%гхв группы от перевала до ночёвки 405 = 6:45



    \FloatBarrier
