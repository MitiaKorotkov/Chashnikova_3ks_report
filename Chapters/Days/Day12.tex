\graphicspath{{Pictures/Chapter5/Day12}}


\subsection{День 12 (11 июля)}\label{subsec:Day12}
    \vbox{%
        \hbox to \textwidth{\hfil%
        \includegraphics[width=0.7\textwidth]{day12.pdf}\label{fig:Day12_map}\hfil}
        \vspace{0.3cm}
        
        \hbox to \textwidth{\hfil%
        \begin{tabular}{|p{4.5cm}|>{\centering\arraybackslash}p{4cm}|}
            \hline
            Расстояние, км		&       \\
            Набор высоты, м		&       \\
            Высота ночёвки, м	&       \\
            Метеоусловия		&       \\
            Покрытие			&       \\
            \hline
        \end{tabular}\quad%
        \begin{tabular}{|p{5cm}|>{\centering\arraybackslash}p{1.5cm}|}
            \hline
                &		\\			
            \hline
                &       \\
                &		\\
            \hline
        \end{tabular}\hfil}%
    }
    \vspace{0.8cm}

подъём 330, выход 521

1) 521 -- 551 (посл 555) набираем воды из снежника, идём вдоль реки

2) 605 -- 655 (посл 659) прошли зачехл часть ледника (налево от ручья), дошли до открытой части ледника Надеваем обвязки, кошки h=3630

3) 729 -- 810 (посл 827) идём в кошках ледник, уходим налево слегка, последние 15м до осыпи тыкаем палкой закрытый ледник и выходим на осыпь
по дороге к туру есть стоянки с водой (почти на перевале)

4) 840 - ? (посл 950) спуск с перевала по осыпи, крупной

Даша ударяет руку, оказываем помощь

5) 1000 -- 1045 (посл 1050) спускаемся пао лифтам и живой осыпи до морены ледника (?) обед h=3500

6) 1205 -- 1230 (?) прошлись в кошках до закрытой части ледника

7) 1240 -- 1315 (посл 1320) идём в связках на следующую ступень ледника. Пересекаем 2 области закрытого ледника. заходим наверх слева пхд

см трек пришли ~18


    \FloatBarrier
