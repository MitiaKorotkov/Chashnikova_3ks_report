\graphicspath{{Pictures/Chapter5/Day12}}


\subsection{День 12 (11 июля)}\label{subsec:Day12}
    \parbox[c]{\textwidth}{%
        \includegraphics[width=0.43\textwidth]{day12.pdf}\label{fig:Day12_map}%
        \raisebox{\height}{\vbox{%
            \hbox to 0.55\textwidth{\hfil%
                \begin{tabular}{|p{4.5cm}|>{\centering\arraybackslash}p{4cm}|}
                    \hline
                    Расстояние, км		&    \\
                    Набор высоты, м		&   \\
                    Высота ночёвки, м	&    \\
    %                Метеоусловия		&       \\
    %                Покрытие			&       \\
                    \hline
                \end{tabular}\hfil}%
            \vspace{0.3cm}

            \hbox to 0.55\textwidth{\hfil%
                \begin{tabular}{|p{5cm}|>{\centering\arraybackslash}p{1.5cm}|}
                    \hline
                        &		\\			
                    \hline
                        &       \\
                        &		\\
                    \hline
                \end{tabular}\hfil}%
        }}%
    }
    \vspace{0.8cm}
\fxnote{Сделать КАРТИНКУ В ТЕКСТЕ! то есть текст должен быть справа от карты в том числе}
подъём 330, выход 521

1) 521 -- 551 (посл 555) набираем воды из снежника, идём вдоль реки

2) 605 -- 655 (посл 659) прошли зачехл часть ледника (налево от ручья), дошли до открытой части ледника Надеваем обвязки, кошки h=3630

3) 729 -- 810 (посл 827) идём в кошках ледник, уходим налево слегка, последние 15м до осыпи тыкаем палкой закрытый ледник и выходим на осыпь
по дороге к туру есть стоянки с водой (почти на перевале)

4) 840 - ? (посл 950) спуск с перевала по осыпи, крупной

Даша ударяет руку, оказываем помощь

5) 1000 -- 1045 (посл 1050) спускаемся пао лифтам и живой осыпи до морены ледника (?) обед h=3500

6) 1205 -- 1230 (?) прошлись в кошках до закрытой части ледника

7) 1240 -- 1315 (посл 1320) идём в связках на следующую ступень ледника. Пересекаем 2 области закрытого ледника. заходим наверх слева пхд

1335 -- 1407 = 32

1412 -- 1450 = 38

1500 -- 1530 = 30

1540 -- 1610 = 30

1620 -- 1640 = 20


чхв 1го до перевала 121 = 2:01
чхв группы до перевала 146 = 2:26
гхв группы до перевала 186 = 3:06


чхв 1го от перевала до ночёвки 100 + (трек) 150 = 250 = 4:10
чхв группы от перевала до ночёвки + 15 + 5 + 0 + 5 = 300 = 5:00
гхв группы от перевала до ночёвки 405 = 6:45



    \FloatBarrier
