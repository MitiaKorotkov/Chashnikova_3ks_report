\graphicspath{Pictures/Chapter5/Day13}


\subsection{День 13 (12 июля)}\label{subsec:Day13}
    \begin{figure}[ht]
        \centering
        \includegraphics[width=0.6\textwidth, angle=90]{Pictures/Chapter5/Day13/day13.pdf}\label{fig:Day13_map}

        \begin{tabular}{|p{4.5cm}|>{\centering\arraybackslash}p{4cm}|}
            \hline
            Расстояние, км		&       \\
            Набор высоты, м		&       \\
            Высота ночёвки, м	&       \\
%            Метеоусловия		&       \\
%            Покрытие			&       \\
            \hline
        \end{tabular}\quad
        \begin{tabular}{|p{5cm}|>{\centering\arraybackslash}p{1.5cm}|}
            \hline
            	&		\\			
            \hline
                &       \\
                &		\\
            \hline
        \end{tabular}
    \end{figure}

подъём 4:30
1) 6:51 -- 7:12 (7:20) h=4100 идём на переал от ночёвки

2) 7:35 -- 8:15 (8:17) h=3785 спуск с перевала по лифтам, есть выходы льда в конце спуска, стоит быть аккуратными

3) 8:30 -- 9:50 выходим на среднюю часть ледника, связываемся, идём закрытую часть. распутываем трещины. разорванную часть (открытую) обходим справа пхд. Далее спуск рядом с водопадом

4) 10:21 -- 11:01 (11:03) h=3135 начинаем спуск справа от всех водопадов по осыпи и снежным мостам

5) 11:21 -- 12:04 (12:08) h=2930 продолжаем спуск по осыпи вдоль ручья

6) 12:17 -- 12:55 (13:01) h=2680 прод спуск по заросшей старой морене, держимся вблизи русла ручья, правый борт. кушаем сухую часть обеда

+1ч чхв дошли до МН, окопались от дождя. стоило идти выше, тогда не на сырости стояли бы


ЧХВ группы: 29 + 42 + 80 + 42 + 47 + 

ЧХВ первого: 21 + 40 + 80 + 40 + 43 + 

    \FloatBarrier
