\graphicspath{{Pictures/Chapter5/Day13}}


\subsection{День 13 (12 июля)}\label{subsec:Day13}
    \vbox{%
        \hbox to \textwidth{\hfil%
        \includegraphics[width=0.7\textwidth]{day13sm.pdf}\label{fig:Day13_map}\hfil}
        \vspace{0.3cm}
        
        \hbox to \textwidth{\hfil%
        \begin{tabular}{|p{4.5cm}|>{\centering\arraybackslash}p{2.8cm}|}
            \hline
            Расстояние, км	&  7.7     \\
            Набор высоты, м	&  +85/-1710     \\
            Высота ночёвки, м	&  2390     \\
    %        Метеоусловия		&       \\
    %        Покрытие			&       \\
            \hline
        \end{tabular}\quad%
        \begin{tabular}{|p{3.5cm}|>{\centering\arraybackslash}p{1.5cm}|}
           \hline
            ЧХВ за день   &	  5:40	\\
            ГХВ за день & 6:55      \\
            \hline
        \end{tabular}\hfil}%
    }
    \vspace{0.8cm}
Утром узнаем прогноз погоды -- обещают дождь во второй половине дня. Предыдущие 2 дня были непростые, накопилась усталость. Решаем не идти на вершину МВТУ и сразу спускаться вниз. В 6:50 выходим, по простому осыпному гребню идем на перевал Шаурту от места ночевки. Хочется отметить исключительно красивые панорамные виды с гребня пер. Шаурту. В 7:25 группа на перевале, фотографируемся. В 7:35 начинаем спуск по лифтовой осыпи плотной группой, есть выходы льда в конце спуска, стоит быть аккуратными.

В 8:30 выходим на ледник Шаурту, связываемся. Зону трещин обходим справа пхд. В 10:20 выходим с ледника на осыпь. При выходе с ледника на осыпь есть небольшая зона бараньих лбов, они некрутые и приятные. Из ледника течет ручей, движемся вниз по его руслу. Частично ручей проходит под снежниками, в том числе движемся и по ним, когда это целесообразно. Левее ручья красивые водопады.

В 12:15  выходим к заросшей старой морене, держимся вблизи русла ручья, правым бортом. Здесь трек уходит на гребень морены, где на картах обозначена, однако мы этот поворот пропустили. Теперь гребень морены отделен от нас крутым конгломератным бортом, подниматься по которому небезопасно, особенно в дождь. С тоской смотрим на заросли зеленого мелколесья в долине рек Тютюргу и Шаурту и перестраиваем маршрут.

Движемся вдоль реки, рельеф достаточно простой. Под проливным дождем в 14:25 доходим до первого приемлемого места ночевки, ставим тент и палатки. Через некоторое время дождь заканчивается, однако мы решаем, что дальше сегодня не пойдем. Сушим вещи и отдыхаем.


%подъём 4:30
%1) 6:51 -- 7:12 (7:20) h=4100 идём на переал от ночёвки

%2) 7:35 -- 8:15 (8:17) h=3785 спуск с перевала по лифтам, есть выходы льда в конце спуска, стоит быть аккуратными

%3) 8:30 -- 9:50 выходим на среднюю часть ледника, связываемся, идём закрытую часть. распутываем трещины. разорванную часть (открытую) обходим справа пхд. Далее спуск рядом с водопадом

%4) 10:21 -- 11:01 (11:03) h=3135 начинаем спуск справа от всех водопадов по осыпи и снежным мостам

%5) 11:21 -- 12:04 (12:08) h=2930 продолжаем спуск по осыпи вдоль ручья

%6) 12:17 -- 12:55 (13:01) h=2680 прод спуск по заросшей старой морене, держимся вблизи русла ручья, правый борт. кушаем сухую часть обеда

%+1ч чхв дошли до МН, окопались от дождя. стоило идти выше, тогда не на сырости стояли бы

%готово?
%чхв группы  5:40
%гхв группы  6:55

%ЧХВ группы: 29 + 42 + 80 + 42 + 47 +

%ЧХВ первого: 21 + 40 + 80 + 40 + 43 +

    \FloatBarrier
