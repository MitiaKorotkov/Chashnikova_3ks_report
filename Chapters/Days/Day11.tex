\graphicspath{{Pictures/Chapter5/Day11}}

\subsection{День 11 (10 июля)}\label{subsec:Day11}
    \vbox{%
        \hbox to \textwidth{\hfil%
        \includegraphics[width=0.8\textwidth]{day11.pdf}\label{fig:Day11_map}\hfil}
        \vspace{0.3cm}
        
        \hbox to \textwidth{\hfil%
        \begin{tabular}{|p{4.5cm}|>{\centering\arraybackslash}p{2.8cm}|}
            \hline
            Расстояние, км		&   $6,7$           \\
            Набор высоты, м		&   $+770 / -710$   \\
            Высота ночёвки, м	&   $3296$          \\
            \hline
        \end{tabular}\quad%
        \begin{tabular}{|p{7cm}|>{\centering\arraybackslash}p{1.5cm}|}
            \hline
            ЧХВ за день &	9:09	\\
            ГХВ за день &	10:28	\\
            \hline
            ЧХВ от МН до пер.\,Столбовой                &   2:22    \\
            ЧХВ от пер.\,Столбовой до пологой долины    &   3:43    \\
            ЧХВ от пологой долины до МН                 &   3:24    \\
            \hline
        \end{tabular}\hfil}%
    }
    \vspace{0.8cm}

    Выходим сегодня рано: вчера был несложный день, в отчетах упоминается, что пер. Столбовой может быть камнеопасным.

    Выход в 4:45. Движемся по огромным полям старого поросшего лишайником курумника. По пути встречаем расчищенные
    небольшие места под палатки, они указаны в файле с фотографиями. В 6:30 выходим в цирк пер.\,Столбовой. Перевальный
    взлет осыпной, крутизна до $30^\circ$, местами осыпь живая и неприятная. В 7:20 собираемся на перевале, седловина
    узкая, есть небольшая площадка под палатку. Если пройти чуть дальше по гребню в сторону вершины 50-летия КБАССР
    можно найти еще один тур и хорошие стоянки. Движемся по гребню в сторону вершины. Скальные выходы обходим справа
    пхд. Собираемся на седловине за вершиной. Здесь к седловине примыкает ледник, который выглядит вполне безопасно.
    Со склонов справа пхд постоянно сыпет камнями.

    В 8:40 готовимся дюльферять. Выход с осыпи на ледник неприятный, осыпь едет, нужно быть очень аккуратными. Для
    подстраховки мы вешаем <<вспомогательную>> веревку с обледенелого снежника на седловине (2 неполностью закрученных
    бура в плохой лед). Станция очень ненадежная, но вес человека с рюкзаком держит.

    Станцию на льду делаем чуть левее пхд, чтобы не сбросить на нее камни. Далее провешиваем еще 4 веревки на ледовом
    склоне крутизной до $35^\circ$. Первые 3 веревки участники захотели снять с нижней страховкой, последние 2
    сдергиваем с проушины. Ледник открытый, на последней веревке ледник закрывается и переходит в снежник. В 11:20
    собираемся в безопасном месте на осыпном островке.

    В 11:50 по осыпи и снежникам спускаемся к ручью и останавливаемся на перекус. Тут неплохое место для ночевки.
    Дальше наш путь идет через огромные поля крупной и средней осыпи. В 14:15 с трудом находим воду, вытекающую с
    грязного ледника, варим суп.

    К 17:15 выходим через моренные поля к отличному месту ночевки. Здесь очень много мест под палатки и течет ручей
    с чистой водой.

    \FloatBarrier
    
    %подъём 3,  выход 445
    
    %1) 445 -- 525 (посл 539) h=3365 идём к цирку пер по курумнику, он для киборгов. облачно, туман
    
    %2) 549 -- 625 (посл 630) h=3535 вышли в цирк перевала. рядом с цирком есть ночёвки со снежником. облака приближаюстя
    
    %3) 640 -- 722 (посл 727) вылезли на перевал, есть мелкая площадка под палатку h=3710
    
    %4) 740 -- .. по дороге на спуск через вершину ещё один тур с хорошими стоянками рядом с туром
    
    %5) 840 собрались дюльферять. 1ая верёвка с 2 буровв снежнике на перевале
    
    %11 -- дюльфернули 5 верёвок (3 -- я на 3 такта, 2 -- проушины)
    
    %1129 -- 1149 (посл 1202) h=3340 спускаем по крутому снежнику до пологой долины
    
    %выход на крутой снежник нетрвиальный (см фото)
    
    %сладкий перекус (1/2 обеда)
    
    %1228 -- 1313 идём моренную помойку / снежники h=3200 (посл 1320)
    
    %1330 -- 1416 дошли до куска ледника, кормим Стефу супом, обед
    
    %1544 -- 1640 h=3230 идём мимо плохих ночевок к хорошим (посл 1644)
    
    %1654 -- 1711 (посл 1715) пришли на ночёвку, вода так себе
    %отбой 1930
    
    %готово
    %чхв 1го до Столбового 118 = 1:58
    %чхв группы до Столбового 142 = 2:22
    %гхв группы до Столбового 162 = 2:42
    
    %чхв 1го от Столбового до пологой долины (неточно!) 200 = 3:20
    %чхв группы от Столбового до пологой долины (неточно!) 223 = 3:43
    %гхв группы от Столбового до пологой долины 262 = 4:22
    
    %чхв 1го от пологой долины до ночёвки = 45 + 46 + 56 + 17 = 164 = 2:44
    %чхв группы от пологой долины до ночёвки 184 = 3:04
    %гхв группы от пологой долины до ночёвки 204 = 3:24 (без учёта супа, да)
    
    %чхв 1го за день = 8:02
    %чхв группы за день = 9:09
    %гхв группы за день = 10:28