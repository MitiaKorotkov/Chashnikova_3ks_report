\graphicspath{{Pictures/Chapter5/Day14}}

\subsection{День 14 (13 июля)}\label{subsec:Day14}
    \vbox{%
        \hbox to \textwidth{\hfil%
        \includegraphics[width=0.7\textwidth]{day14sm.pdf}\label{fig:Day14_map}\hfil}
        \vspace{0.3cm}
        
        \hbox to \textwidth{\hfil%
        \begin{tabular}{|p{4.5cm}|>{\centering\arraybackslash}p{2.8cm}|}
            \hline
            Расстояние, км		&   $17,6$          \\
            Набор высоты, м		&   $+250 / -540$   \\
            Высота ночёвки, м	&   $2104$          \\
            \hline
        \end{tabular}\quad%
        \begin{tabular}{|p{6cm}|>{\centering\arraybackslash}p{1.5cm}|}
            \hline
            ЧХВ за день & 4:41	\\
            ГХВ за день & 5:41	\\
            \hline
            ЧХВ от МН до заброски    &   3:20   \\
            ЧХВ от заброски до МН    &   1:21	\\
            \hline
        \end{tabular}\hfil}%
    }
    \vspace{0.8cm}

    Выходим из лагеря в 6:20, доходим до мелколесья в разливах рек Шаурту и Тютюргу, лес на удивление достаточно
    проходимый, продираться сквозь заросли не приходится. В 7:20 переходим мост и выходим на хорошую проезжую дорогу.
    Места здесь очень приятные: чистый сосновый лес, есть обустроенные костровища, скамеечки. Жаль, что не дошли
    сюда вчера. Проходим заброшенную турбазу Чегем.

    В 9:20 подошли к мосту через реку Гара-Аузу-Су, к 10:30 дошли до водопада Абай-Су. Мы правильно выбрали дорогу
    по левому орографически борту реки Гара-Аузу-Су, на противоположном берегу достаточно оживленное движение.
    Водитель привозит нам заброску, сушим вещи, раскидываем продукты, распугиваем туристов, приехавших посмотреть
    на водопад.

    В 13:45 собравшись, выходим дальше к турбазе Башиль. Идём по дороге, переходящей в тропу. Идем по полям и лесам,
    долина достаточно живописна. Ночуем в приятном сосновом лесу, ровных мест под палатки не так уж много, кочки.
    Вода в реке очень мутная, за водой сходили к роднику, набрали впрок.

    \FloatBarrier
    
    %подъём 4:20 факт
    %1) 6:23 -- 7:07 (7:10) идём к лесу  и по осыпи в лесу (??) лес проходимый, привал у р тютюргу h=2260
    
    %2) 7:21 -- 8:12 h=2135 перешли мост через тютюргу вышли на (проезжую) дорогу, прошли какую-то турбазу
    
    %3) 8:21 -- 9:20  h=1895 дошли до моста через Гара-***?
    
    %4) 9:36 -- 10:05
    
    %5) 10:18 -- 10:32
    %дошли до водопада (место заброски) h=1935
    
    %водитель привёз заброску в нужное место и время
    
    %согласился увезти наш мусор
    
    %обед, сиеста, разложение, сушка
    
    %6) 13:45 -- 14:26 h=2025
    %идём по дороге, переходящей в тропу. во второй половине ходки -- по лесу
    
    %7) 14:36 -- 15:16 дошли до рандомной точки в лесу, назвали ночёвкой, красивые сосны. нашли подкову. скушали ништяки
    
    %забрались на сосну
    
    %чхв 1го от ночёвки до заброски и обеда: 197 = 3:17
    %чхв группы от ночёвки до заброски и обеда: 3:20
    %гхв группы от ночёвки до заброски и обеда: 250 = 4:10
    
    
    %ЧХВ 1го за день: 4:38
    %ЧХВ группы за день: 4:41
    %ГХВ группы за день: 5:41
    %готово