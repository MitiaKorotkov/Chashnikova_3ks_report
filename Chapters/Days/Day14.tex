\graphicspath{{Pictures/Chapter5/Day14}}


\subsection{День 14 (13 июля)}\label{subsec:Day14}
    \vbox{%
        \hbox to \textwidth{\hfil%
        \includegraphics[width=0.7\textwidth]{day14.pdf}\label{fig:Day14_map}\hfil}
        \vspace{0.3cm}
        
        \hbox to \textwidth{\hfil%
        \begin{tabular}{|p{4.5cm}|>{\centering\arraybackslash}p{4cm}|}
            \hline
            Расстояние, км		&       \\
            Набор высоты, м		&       \\
            Высота ночёвки, м	&       \\
            Метеоусловия		&       \\
            Покрытие			&       \\
            \hline
        \end{tabular}\quad%
        \begin{tabular}{|p{5cm}|>{\centering\arraybackslash}p{1.5cm}|}
            \hline
                &		\\			
            \hline
                &       \\
                &		\\
            \hline
        \end{tabular}\hfil}%
    }
    \vspace{0.8cm}

подъём 4:20 факт
1) 6:23 -- 7:07 (7:10) идём к лесу  и по осыпи в лесу (??) лес проходимый, привал у р тютюргу h=2260

2) 7:21 -- 8:12 h=2135 перешли мост через тютюргу вышли на (проезжую) дорогу, прошли какую-то турбазу

3) 8:21 -- 9:20  h=1895 дошли до моста через Гара-***?

4) 9:36 -- 10:05 

5) 10:18 -- 10:32 
дошли до водопада (место заброски) h=1935 

водитель привёз заброску в нужное место и время

согласился увезти наш мусор

обед, сиеста, разложение, сушка

6) 13:45 -- 14:26 h=2025
идём по дороге, переходящей в тропу. во второй половине ходки -- по лесу

7) 14:36 -- 15:16 дошли до рандомной точки в лесу, назвали ночёвкой, красивые сосны. нашли подкову. скушали ништяки

забрались на сосну


ЧХВ группы: 47 + 51 + 59 + 29 + 14 + 41 + 40 = 4:41

ЧХВ первого: 44 + 51 + 59 + 29 + 14 + 41 + 40 = 4:38

    \FloatBarrier
