\graphicspath{{Pictures/Chapter5/Day7}}


\subsection{День 7 (6 июля)}\label{subsec:Day7}
    \vbox{%
        \hbox to \textwidth{\hfil%
        \includegraphics[width=0.7\textwidth]{day7.pdf}\label{fig:Day7_map}\hfil}
        \vspace{0.3cm}

        \hbox to \textwidth{\hfil%
        \begin{tabular}{|p{4.5cm}|>{\centering\arraybackslash}p{3cm}|}
            \hline
            Расстояние, км		&   1.5    \\
            Набор высоты, м		&   +350/-270    \\
            Высота ночёвки, м	&      4168 \\
            \hline
        \end{tabular}\quad%
        \begin{tabular}{|p{7cm}|>{\centering\arraybackslash}p{1.5cm}|}
            \hline
              ЧХВ за день:  &    10:05	\\		
              ГХВ за день:   &   11:50	\\		
            \hline
              ЧХВ от МН до п. Башхаузбаши  & 5:30      \\
              ЧХВ от п. Башхаузбаши до МН  & 4:35      \\
            \hline
        \end{tabular}\hfil}%
    }
    \vspace{0.8cm}

Выход группы ранний, в 5:25, т.к. по прогнозу во второй половине дня дождь и снег.

Выходим по ЮВ гребню в сторону вершины Башхауз. Сначала гребень простой -- широкий, осыпной, уклон до $15^\circ$. Первые выходы скал обходим справа пхд.
Дальше гребень усложняется, кроме осыпи встречаются участки разрушенных скал, простое лазанье. Связки не делаем -- очень много живых камней, которые можно сдернуть веревкой. Особые эмоции доставил короткий узкий участок скал, шириной и длиной около метра, который нужно было преодолевать максимально аккуратно.
Большие скалы, разделяющие ЮВ и Ю гребень проходим справа пхд по осыпному склону с  выходами разрушенных скал (до $30^\circ$ крутизной). При подъеме для обхода выходов скал приходится пересекать снежники -- для этого организуем перильную страховку на траверсе. Осыпь под снежниками мокрая и сильно едет, края снежников и подложка -- ледяные легко поскользнутся в ботинках. Станции делаем на скальных выступах, но найти хорошие выступы очень сложно, скалы сильно разрушены.

В 6:30 собираемся на широком участке перед кулуаром на высоте 4255. Удобных мест для отдыха и сбора группы не так много. В 7:40 подошли к началу подъемного кулуара и организовали станцию. Станция на выступе левее пхд кулуара, немного не доходя на него. Сидим под скалами, в случае чего можно спрятаться от летящих камней.

Подъем по кулуару до $45^\circ$ крутизной, достаточно много выступов, лазанье не сложное. Основная опасность -- много живых достаточно крупных камней, за которые можно случайно ухватиться и достать на себя. Нам повезло с грамотным техлидером: он провешивает перила и дополнительно чистит кулуар. После подъема небольшой участок до 50 м проходим без веревки по осыпи левее и вверх пхд до следующей станции. Провешиваем около 15 м перил от ЮВ до Ю кулуара. Отсюда за 25 минут 2 наиболее опытных и быстрых участника поднимаются на вершину по альпинистской тропе и возвращаются обратно, поменяв записку. На вершине они были в 12:30, взяли с собой рацию, регулярно связываемся по ней. Руководитель остается контролировать прохождение группой траверса.

В 12:00 небо уже закрывали тучи и от ясной солнечной погоды не осталось и следа. Учитывая, что до первого хорошего места ночевки необходимо еще и спуститься с перевала МВТУ, на вершину группой не идем. Начинаем спуск по альпинистской тропе на пер. МВТУ. Для ускорения спуска остаток веревки провесили ниже вдоль гребня. На тропе также нужно идти аккуратно, но намного  проще, чем по ЮВ гребню. Видимость ухудшается, начинается снег. Руководитель и девочки уходят вперед, мальчики снимают веревки, связь держим по рации. Идем в грозовом облаке, волосы на бровях и ресницах шевелятся, видим небольшие разряды молний, которые ударяют, кажется, совсем близко. Когда становится совсем страшно, снимаем все железо, оставляем рюкзаки под накидками и ледорубы и прячемся в относительно пологом месте под тентом, чтобы переждать. Через минут 20 к нам присоединяются мальчики, еще минут 30 сидим под тентом, перекусываем. Погода не меняется, ночевка все еще неблизко. Собираем тент и идем дальше в сторону перевала МВТУ. В 15:20 собираемся на перевале МВТУ, находим расходник и камень, с которого спускалась, встреченная нами 2 дня назад группа. Выбираем кулуар для спуска - он полностью снежный от седловины, риск скинуть камни вниз с осыпи здесь меньше. Провешиваем первую веревку на дюльфер, нижнюю станцию делаем в снегу на ледорубе. Немного не хватило до пологой части ледника. Провешиваем вторую веревку для подстраховки от проваливания в трещины, вторая станция на ледобуре, ледник совсем пологий. Вторую веревку проходим на пруссике и скользящем карабине. В 17:30 дюльфернули, собираемся у второй станции, вяжем связки.

Пробуем пройти дальше, но участник проваливается в трещину с головой. Пока мы закручивали бур, он успел вылезти сам. Возвращаемся назад и встаём на ночёвку в 17:45 на леднике, в цирке перевала МВТУ. Погода улучшилась, открылись виды на пятитысячники ГКХ.



%Подъём 330
%Выход 522 -- 525

%Около 543 обходим справа пхд выходы скал
%~615 проходим след скальный выход. есть участок "нож" <=2м, чистим скалы от живых камней.

%6:21 (посл 6:29) собираемся на широком участке перед кулуаром h=4255

%639...740 плотно подошли к кулуару

%753 повесили верёвку  через кулуар слева направо пхд

%1730 дюльфернули с перевала на ледник; 2000 отбой

%Пробуем пройти дальше, но Лёша попадает в трещину. забиваем, ставимся на леднике



    \FloatBarrier
