\section{Вводная часть и полезные сведения}\label{sec:introduction}
	\subsection{Идея похода, выбор района и логика построения маршрута}
		Целью похода было пройти спортивный, насыщенный разнообразными техническими препятствиями маршрут 3\,к.с.
		в достаточно красивом районе. Можно считать цель выполненной~"--- группа познакомилась с ледопадом на
		пер.~Спартак, потренировала гребневое лазание на траверсе вер.~Башхауз и вер.~КенчатБаши, отработали протяженный
		спуск по ледовому склону на пер.~Столбовой, потренировались проходить скальный рельеф на подъем на траверсе
		вер.~Башхауз и на пер.~Шаурту. Маршрут получился очень красивым, вдоволь насладились панорамами вершин на
		рассветах и закатах. Немного подвела погода на спуске по Безенгийскому леднику~"--- с самого утра зарядило молоко
		и верхнюю часть стены вблизи так и не увидели. Таким образом, район Безенгов и Адырсу отлично подходит для
		проведения горных походов 3\,к.с. 
		
		При планировании маршрута мы учитывали, что в группе были студенты, которые защитили диплом за несколько дней
		до похода и которым требовался отдых. Также нам нужна была мягкая акклиматизация, поэтому первые два дня мы
		ходили в радиальный выход на вершину Штулу-Тау. Далее маршрут был логично разделен на 3 кольца, максимальная
		продолжительность кольца вместе с днем запаса~"--- 7 дней. Мы старались не брать лишнего в личное снаряжение,
		в результате при желании можно было собрать рюкзак в 17--20\,кг для девушек и 21--25 для юношей. Гендерная разница
		по весу общественного снаряжения и еды была фиксированной и составляла 3\,кг. Поход был запланировани на начало
		июля, так как в это время в районе достаточно снега и прохождение препятствий более безопасно. В частности, в
		нитке маршрута были перевалы, которые к августу могут стать камнеопасными: пер.~Туристов Грузии, Столбовой, Кенчат.
	
	
	\subsection{Предпоходная подготовка}
		В рамках подготовки к походу участники самостоятельно тренировали ОФП (1\,ч кардио 2 раза в неделю), проводились
		совместные технические тренировки в Полушкино, по итогам которых в начале июня группа поучаствовала в Кросс-Походе
		Вестры. Также было несколько совместных выходов в формате ПВД. Желающие посещали лекции в горном клубе МГУ, но это
		не контролировалось руководителем.
	

	\subsection{Логистика и взаимодействие с различными службами}
		\subsubsection{Транспорт, заброски и проживание}

			Забрасывались и выезжали с маршрута мы через Минеральные Воды.
			
			Насчёт трансфера мы договорились с \dots

			Таблица с собранными контактами:
			\setlength{\arrayrulewidth}{0.2mm}
			\setlength{\tabcolsep}{0pt}
			\renewcommand{\arraystretch}{1}
			\begin{longtable}{|>{\centering\arraybackslash}m{3cm}|>{\centering\arraybackslash}m{2cm}|>{\centering\arraybackslash}m{4cm}|>{\centering\arraybackslash}m{4cm}|>{\centering\arraybackslash}m{5cm}|} \hline
				Т/б <<Глобус>> 	&											&	+7(989)786-51-00 +7(928)759-05-85	&	Махар-Биджиевы		&	обычной связи на базе нет, с хозяевами нужно связываться по WhatsApp                	\\ \hline
				Трансфер 		&	\href{https://transfer-tur.ru}{сайт}	&	+7(928)409-55-12					&	Гофман Михаил		&	Пользовались их услугами при заброске и возвращении                                 	\\ \hline
				Трансфер 		&	\href{https://transfer-tur.ru}{сайт}	&	+7(918)681-55-45					&	Гофман Сергей		&	Пользовались их услугами при заброске и возвращении                                 	\\ \hline
				Трансфер 		&	\href{http://minvodytransfer.ru}{сайт}	&	+7(906)462-21-55					&	Николаев Андрей		&	Есть группа в ВК                                                                    	\\ \hline
				Трансфер 		& 											&	+7(928)950-38-68					&	Саракуев Борис		&	С ними ездила группа Крохалева Олега.                                               	\\ \hline
				Трансфер 		& 											&	+7(928)394-75-01					&	Алимурат			&	Отвозил наши Заброски в "Глобус" и "Узункол" в походе Кавказ-2023                   	\\ \hline
				Трансфер 		& 											&	+7(925)325-57-69					&	Амир				&	УАЗик, заброска на <<Глобус>>, <<Узункол>>                                          	\\ \hline
				Трансфер 		&	Махар	 								&	+7(938)653-46-66					&	Мурат				&	УАЗик, заброска на <<Глобус>>, <<Узункол>>                                          	\\ \hline
				Дом		 		&	Хурзук	 								&	+7(989)784-40-30 +7(928)395-19-54	&	Мухамммат (Михаил)	&	Гостевой домик в Хурзуке.                                                           	\\ \hline
				Трансфер 		&	Узункол	 								&	+7(938)025-70-05					&	Аберек				&	Покупали у него хычины. Есть УАЗик, может сделать заброску в <<Глобус>>, <<Узункол>>	\\ \hline
			\end{longtable}\fxnote{Поправить или выкинуть}
			\setlength{\arrayrulewidth}{0.3mm}
			\setlength{\arraycolsep}{5pt}
			\renewcommand{\arraystretch}{1.8}

		\subsubsection{Погранзона}
			Небольшая часть нашего маршрута проходила внутри пятикилометровой погранзоны. Поэтому мы оформили
			погранпропуска. Мы решили оформлять индивидуальные пропуска. Они были оформлены по почте. 
			
			В этом году у пограничников сменилась форма заявления, поэтому вариант с его подачей через ГосУслуги
			не работал. Причём на сайте фсб висели старые бланки, которые принимать у нас отказались, выслав в
			ответном письме правильные.
			
			Заполненные заявления мы выслали по электронной почте pu.kcherkes@fsb.ru за 2,5 месяца до начала похода.
			Примерно через две недели участники получили письма с оформленными пропусками. 
			
			На маршруте погранпропуска у нас проверяли при выходе из погранзоны на кпп в д.р. Махар чуть выше
			т/б <<Глобус>> (в погранзону мы зашли через долину реки Кичкинакол, где кпп нет). Также пропуска
			у нас смотрели при выходе из долины реки Кубань около аула Актюбе. Оба раза вместе  пропусками у
			нас просили показать и паспорта.
			\fxnote{Переписать}


		\subsubsection{МЧС}
			Группа зарегистрировалась в МЧС на \href{https://forms.mchs.gov.ru/registration_tourist_groups}{сайте}
			до начала похода. По прибытии на точку старта группа сообщила о начале маршрута в ПСО по телефону
			$+7 (878) 226-60-56$. Связь с МЧС поддерживалась на протяжении всего похода согласно графику, указанному
			на сайте при регистрации. По завершении маршрута группа также сообщила об этом в ПСО.
			\fxnote{Переписать, поправить номера и ссылки}

		\subsubsection{Страховка}
			Страховку оформляли в Совкомбанке, программа <<Стандарт>> (включает эвакуацию вертолетом), тип поездки
			<<Спорт>> с суммой покрытия 5000 у.е. на каждого. Трое человек из группы не захотели оформлять страховку.
 
			\textit{Определить тип поездки было затруднительно, так как нигде явно не прописан спортивный туризм.
			Кроме <<Спорта>> возможным вариантом был <<Экстремальный отдых>>, страхование этого типа поездки в
			несколько раз дороже. На момент оформления полиса под <<Экстремальный отдых>> попадали такие активности,
			как маунтибайки, банджи-джампинги, альпинизм и т.д. Под <<Спорт>> попадали все спортивные занятия и
			участия в соревнованиях, не перечисленные в <<Экстремальном отдыхе>>. Так что мы решили, что
			<<Спорт>> -- для нас. Однако уже пару месяцев спустя (на момент написания отчета), в список
			<<Экстремального отдыха>> был включен треккинг на высоте больше 1500 метров, и теперь лучше связываться
			с менеджерами для уточнения того, что такое спортивный туризм.}
			\fxnote{Переписать, поправить номера и ссылки}

		\subsubsection{Средства связи}
			

	\subsection{Аварийные и запасные варианты маршрута}
		Для повышения безопасности на маршруте были предусмотрены запасные (для потенциально технически сложных участков) и 
		аварийные варианты пути  на случай ЧП.

		\subsubsection*{Аварийные выходы:}
			\begin{itemize}
				\item д.р. Кору
				\item д.р. Тютюнсу и д.р.
			\end{itemize}

			Эти варианты были заявленны в план-графике.

		\settowidth{\tmplen}{10 день: }
		\addtolength{\tmplen}{\leftmargin}
		\subsubsection*{Запасные варианты:}
			\begin{enumerate}[leftmargin=\tmplen]
				\item[\uline{10 день}:] в случае камнеопасности спуска с пер. Столбовой пройти дальше по гребню на север и спускаться с осыпной седловины между вершинами Сарыкая и Каргашиль.
				\item[\uline{11 день}:] выход по д.р. Кору в пос. Булунгу - вариант экономии 2 дней при отставании от плана графика.
				\item[\uline{11 день}:] пер. Тютюргу Зап. (1А) - т/б Чегем - вариант экономии 1 дня при отставании от плана графика.
				\item[\uline{15 день}:] пер. Надежда + пер. Килар (1Б) - отказ от прохождения пер. Кенчат (2А), в случае его камнеопасности, обход через более простые перевалы.
				\item[\uline{15 день}:] обход по д.р. Кенчат + пер. Килар (1Б) - отказ от прохождения пер. Кенчат (2А) и пер. Надежда (1Б, п/п), в случае камнеопасности, обход через долины рек.
				\item[\uline{17 день}:] выход по д.р. Тютю-су - экономия 1-2 дней при отставании от плана графика.
			\end{enumerate}
\fxnote{Сделать хорошо оформление заголовков аварийные выходы и запасные варианты}

%		\subsubsection{Аварийные варианты}
% 			\begin{enumerate}
%				\item \dots
%			\end{enumerate}
